\chapter[伸縮量制御による振幅調整]%
{伸縮量制御による振幅調整}
        \section{はじめに}

        ロバストなブラキエーション動作を行うために,
        本章では,伸縮量を制御することによる振幅調整システムについて述べる.

        \chapref{chapter4}において導出した最適なバーリリース条件を実現するためには,
        振子過程における振幅調整が望まれる.
        本章では伸縮タイミングと伸縮量の調整による目標角度・角速度の実現について述べる.
          
        \section{伸縮による励振のメカニズム}

          本研究では,重心を移動させることにより振子過程において励振させる.
          Lieskovsk{\`y}らの重りを動かすことによる振幅増加率が最大となる重心移動の研究\cite{lieskovsky2023optimal}を,
          伸縮機構に応用して励振を行う.
          
          \subsection{伸縮する単リンクブラキエーションロボットのモデル化}
          \figref{modelFig.eps}に伸縮する単リンクブラキエーションロボットを簡略化した,
          伸縮する振子のモデルを示す.振子の半径方向と鉛直下向き線がなす角を$\varphi [\mathrm{rad}]$,
          振子の長さを$l [\mathrm{m}]$,振子の質量を$m [\mathrm{kg}]$,振子の重心周りの慣性モーメントを$I [\mathrm{kg}\mathrm{m}^2]$,
          重力加速度を$g [\mathrm{m}/\mathrm{s}^2]$とする.
          \fig{modelFig.eps}{width=0.3\hsize}{Model of an extensible pendulum}
          運動方程式を以下のラグランジュの運動方程式で求める.
          \begin{eqnarray}
            \equlabel{Lagrange}
            \frac{\mathrm{d}}{\mathrm{d}t}\left(\frac{\partial L}{\partial \dot{q_i}}\right)-\frac{\partial L}{\partial q_i}=S_i
            \end{eqnarray} 
          $q_i$と$S_i$はそれぞれ一般化座標と一般化力であり,それぞれ\equref{qq},\equref{SS}とした.
          ここで,伸縮のために加える力を$u[\mathrm{N}]$とする.
          $L=T-U$はラグランジアンであり,
          系の運動エネルギー$T$と位置エネルギー$U$で構成され,本モデルでは\equref{Tequ},\equref{U_equ}となる.
          \begin{eqnarray}
              \equlabel{qq}
              \begin{bmatrix}
                q_1 \\
                q_2 \\
                \end{bmatrix}
              &=&
              \begin{bmatrix}
                \varphi \\
                l \\
                \end{bmatrix}\\
              \equlabel{SS}
              \begin{bmatrix}
                S_1 \\
                S_2 \\
                \end{bmatrix}
              &=&
              \begin{bmatrix}
                0 \\
                u \\
                \end{bmatrix}
                \end{eqnarray}
              \begin{eqnarray}
              \equlabel{Tequ}
              T &=& \frac{1}{2} m \left(\frac{l}{2} \dot{\varphi} \right)^2 + \frac{1}{2} I \dot{\varphi}^2 + \frac{1}{2} m \dot{l}^2  \\
              \equlabel{U_equ}
              U &=& -gm\frac{l}{2}\cos(\varphi)
              \end{eqnarray}  
          運動方程式を行列で書き表すと\equref{Matrix}となる.
          \begin{eqnarray}
            \equlabel{Matrix}
            \underbrace{\frac{\partial^2 L}{\partial \dot{\bm{q}}^2}}_{\bm{M}} \ddot{\bm{q}}^2 +\underbrace{\frac{\partial^2 L}{\partial \dot{\bm{q}} \partial \bm{q}} \dot{\bm{q}} - \frac{\partial L}{\partial \bm{q}}}_{\bm{c}} = 
            \begin{bmatrix}
              0 \\
              u \\
              \end{bmatrix}
            \end{eqnarray} 
          ここで,$\bm{M}(l)$,$\bm{c}(\varphi,l,\dot{\varphi},\dot{l})$は以下のようになる.
            \begin{eqnarray}
              \equlabel{M}
              \bm{M}(l)&=&
              \begin{bmatrix}
                M_{11}(l) & 0\\
                0 & M_{22} \\
                \end{bmatrix}\\
              \equlabel{C}
              \bm{c}(\varphi,l,\dot{\varphi},\dot{l})&=&
              \begin{bmatrix}
                c_{1}(\varphi,l,\dot{\varphi},\dot{l}) \\
                c_{2}(\varphi,l,\dot{\varphi}) \\
                \end{bmatrix}\\
              \equlabel{M11}
              M_{11}(l) &=& \frac{1}{4}m{l(t)}^2 + I\\
              \equlabel{M22}
              M_{22} &=& \frac{m}{4}\\
              \equlabel{C1}
              c_{1}(\varphi,l,\dot{\varphi},\dot{l}) &=& \frac{1}{2}ml(t)\dot{l}(t)\dot{\varphi}(t) + \frac{1}{2}gml(t)\sin{(\varphi(t))}\\
              \equlabel{C2}
              c_{2}(\varphi,l,\dot{\varphi}) &=& -\frac{1}{4}ml(t){\dot{\varphi}(t)}^2 - \frac{1}{2}gm\cos{(\varphi(t))}
            \end{eqnarray} 
          本モデルの力積は\equref{Si}で計算され,系の運動量$p_i = \frac{\partial L}{\partial \dot{q}_i}$との関係は\equref{pi}である.
          \begin{eqnarray}
            \equlabel{Si}
            \hat{S}_i = \lim_{t^{+}\to t} \int_{t}^{t^{+}} S_i (\tau) \mathrm{d}\tau\\
            \equlabel{pi}
            p_i(t^{+})-p_i(t)=\hat{S}_i
            \end{eqnarray}
          本モデルにおいて,$\varphi$方向の運動量は保存されるため,\equref{impulse}が満たされる.
          \begin{eqnarray}
            \equlabel{impulse}
            M_{11}(l^{+})\dot{\varphi}^{+}-M_{11}(l)\dot{\varphi}=0
            \end{eqnarray}
          また,状態変数を
          \begin{eqnarray}
            \bm{x}=
            \begin{pmatrix}
              x_1\\
              x_2\\
              x_3\\
              x_4\\
              \end{pmatrix}
              =
              \begin{pmatrix}
                \varphi\\
                l\\
                \dot{\varphi}\\
                \dot{l}\\
                \end{pmatrix}
            \end{eqnarray}
          とすると,状態モデルは
          \begin{eqnarray}
            \dot{\bm{x}}=
            \begin{pmatrix}
              \dot{x}_1\\
              \dot{x}_2\\
              \dot{x}_3\\
              \dot{x}_4\\
              \end{pmatrix}
              =
              \begin{pmatrix}
                x_3\\
                x_4\\
                -M_{11}^{-1}(x_2)c_1(x_{1-4})\\
                -M_{22}^{-1}c_2(x_{1-3})+M_{22}^{-1}u\\
                \end{pmatrix}
            \end{eqnarray}
          となる.

          \subsection{励振シミュレーション}
          伸縮が時刻$t$秒から$t^{+}$秒の間に瞬間的に行われると仮定すると,
          伸縮後の状態$\bm{x}^{+}$は伸縮前の状態$\bm{x}$と\equref{impulse}を用いて
          \equref{xx}で表される.ここで,伸びる場合は縮む場合は$x_2^{+}=l_{\mathrm{min}}$,$x_2^{+}=l_{\mathrm{max}}$
          となり,\equref{l_change}の条件に基づいて変化させる.
          \begin{eqnarray}
            \equlabel{xx}
            \bm{x}^{+}&=&
            \begin{pmatrix}
                x_1\\
                x_2^{+}\\
                M_{11}^{-1}(x_2^{+})M_{11}(x_2)x_3\\
                0\\
                \end{pmatrix}\\
            \equlabel{l_change}
           l&=& \left\{
              \begin{aligned}
                l_{\mathrm{min}}\quad \mathrm{if}\,\varphi\dot{\varphi}>0\\
                l_{\mathrm{max}}\quad \mathrm{if}\,\varphi\dot{\varphi}<0
              \end{aligned}
            \right.
            \end{eqnarray}
          この瞬間的な伸縮により系の運動エネルギー$T$と位置エネルギー$U$は次のように変化する.
          \begin{eqnarray}
            \equlabel{dT}
            \Delta T_t^{t^{+}}&=&\left(\frac{M_{11}(x_2)}{M_{11}(x_2^{+})}-1\right)\frac{1}{2}M_{11}(x_2)x_3^2\\
            \equlabel{dU}
            \Delta U_t^{t^{+}}&=&gm(x_2^{+}-x_2)\cos{(x_1)}            
            \end{eqnarray}
          本実験で用いる実機に基づき,各パラメータと初期状態$(\varphi_{\mathrm{ini}},l_{\mathrm{ini}},\dot{\varphi}_{\mathrm{ini}},\dot{l}_{\mathrm{ini}})$
          を\tabref{parameter}としてルンゲクッタ法を用いてシミュレータを作成した.
          その結果を\figref{ExcitationSimulation.eps}に示す.
          横軸は時間$t$,縦軸はそれぞれ角度$\varphi$・角速度$\dot{\varphi}$・
          力学的エネルギー$E$・運動エネルギー$T$・ポテンシャルエネルギー$U$を示す.
          これにより,伸縮することによる重心移動でも励振が可能であることが確認できる.
          これらの内容はLieskovsk{\`y}らのおもりを動かすことによる励振の研究\cite{lieskovsky2023optimal}を
          伸縮機構に応用したものである.

          また,伸縮時の最大長$l_{\mathrm{max}}$を0.59 mから0.74 mまで3 cm刻みで
          変化させた場合の励振シミュレーションを\figref{changingLengthExcitation.eps}に示す.
          横軸は時間,縦軸は力学的エネルギーと角度を示している.
          このシミュレーション結果から,伸縮量を大きくするほど振幅増加量が大きくなり,伸縮量を小さくするほど伸縮増加量が小さくなることが分かる.
          ゆえに,伸縮量調整によって励振時の振幅操作が可能であると考えられる.
          \begin{table}[tb]
            \begin{center}
              \caption{Simulation parameters}
              \tablabel{parameter}
              \vspace{2mm}
              \begin{tabular}{c|c}
                \hline
                Variables & Values \\
                \hline
                $m$ [kg] & 3.0 \\
                $l_{\mathrm{min}}$ [m] & 0.56 \\
                $l_{\mathrm{max}}$ [m] & 0.74 \\
                $g$ $\mathrm{[m/s^2]}$ & 9.81 \\
                $\varphi_{\mathrm{ini}}$ [rad] & 0.3 \\
                $l_{\mathrm{ini}}$ [m]& $l_{\mathrm{max}}$ \\
                $\dot{\varphi}_{\mathrm{ini}}$ & 0.0 \\
                $\dot{l}_{\mathrm{ini}}$ & 0.0 \\
                \hline
              \end{tabular}
            \end{center}
          \end{table}
          \fig{ExcitationSimulation.eps}{width=1\hsize}{Excitation simulation of extensible brachiation robot}
          \fig{changingLengthExcitation.eps}{width=1.0\hsize}{Excitation simulation with changing the max length}
        
          
        \newpage
        \section{励振調整の流れ}

          本研究では振幅を基に伸縮量を制御し,励振調整を行う.
          以下に励振調整の流れを示す.
          \begin{enumerate}
            \item 目標振幅を
            \item IMUを用いてロボットの角度・角速度を計測
            \item 角速度の正負が入れ替わる時の角度を現在振幅とする
            \item 現在振幅に基づく伸縮量制御
            \begin{enumerate}
              \item 現在振幅が目標振幅から遠い場合
              \begin{enumerate}
              \item 伸縮可能最大長で励振
              \end{enumerate}
              \item 現在振幅が目標振幅に近い場合
              \begin{enumerate}
                \item 現在振幅と目標振幅の比に基づいて伸縮量を制御
              \end{enumerate}
            \end{enumerate}
            \item 目標振幅に到達したら伸縮量制御を終了
          \end{enumerate}

          \section{伸縮量制御}

          リアルタイムで伸縮量調整を行うために,半周期後の振幅の変化率と伸縮量の関係式を求める.
          実機の振動を,振幅が微小であるとみなした近似式である\equref{Approximation Model}に示す形式での近似モデル化を試みた.
          ここで現在振幅,振幅増加率,角振動数をそれぞれ$A_{\mathrm{now}}$,$\lambda$,$\omega$として
          $t$秒後の変位$\varphi(t)$を表す.
          また,半周期後の振幅を$A_{\mathrm{next}}$とすると,\equref{amp rate}となる.
          \begin{eqnarray}
            \equlabel{Approximation Model}
            \varphi(t)&=&A_{\mathrm{now}}e^{\lambda t}\cos{(\omega t)}\\
            \equlabel{amp rate}
            A_{\mathrm{next}}&=&\left|\varphi\left(\frac{\pi}{\omega}\right)\right|=A_{\mathrm{now}}e^{\lambda \frac{\pi}{\omega}}
          \end{eqnarray}
          よって,現在振幅$A_{\mathrm{now}}$から半周期後に目標振幅$A_{\mathrm{ref}}$になるために必要な減衰率は\equref{lambda}で表される.
          \begin{eqnarray}
            \equlabel{lambda}
            \lambda=\frac{\omega}{\pi}\ln{\left(\frac{A_{\mathrm{ref}}}{A_{\mathrm{now}}}\right)}
          \end{eqnarray}

        \subsection{近似モデルの励振データへのフィッティング}
          
          実機の励振データの取得のために励振計測を行った.
          伸縮時の最小長は0.56 mで固定し,
          最大長のみ0.74 m,0.70 m,0.68 m,0.66 mと変化させた.


          ここで,近似モデルをフィッティングさせるために,
          なお,微小角近似ができない振幅になると角振動数は小さくなるため,
          
          単純化のために振幅で区間を分け,
          区間ごとに角振動数を調整することで近似モデルを実験データにフィッティングさせた.
          \figref{}に計測データとフィッティング結果,\tabref{ExcitationParameters}に振幅の区間分けとそれぞれの区間における
          固有角振動数,\tabref{ExcitationRate}に最大長ごとの減衰率を示す.
          \begin{table}[tb]
            \begin{center}
              \caption{Excitation fitting values}
              \tablabel{ExcitationParameters}
              \vspace{2mm}
              \begin{tabular}{c|c}
                \hline
                Amplitude $A_{\mathrm{now}}$ [deg] & $\omega$ [rad/s]\\
                \hline
                $A_{\mathrm{now}}\ge90$ & 3.35 \\
                $90>A_{\mathrm{now}}\ge85$ & 3.50 \\
                $85>A_{\mathrm{now}}\ge70$ & 3.70 \\
                $70>A_{\mathrm{now}}$ & 3.90 \\                   
                \hline
              \end{tabular}
            \end{center}
          \end{table}
          \begin{table}[tb]
            \begin{center}
              \caption{Excitation rate}
              \tablabel{ExcitationRate}
              \vspace{2mm}
              \begin{tabular}{c|c}
                \hline
                Length $l$ [m] & $\lambda$ [1/s] \\
                \hline
                0.74 & -0.15 \\
                0.72 & -0.14 \\
                0.70 & -0.13 \\
                0.68 & -0.090 \\
                0.66 & -0.0010 \\                     
                \hline
              \end{tabular}
            \end{center}
          \end{table}
          \fig{photo.jpg}{width=1\hsize}{Excitation experiment data and fitting data}
        \figref{}に示すように,縦軸を減衰率,横軸を最大長さとして\tabref{ExcitationRate}のデータをプロットし,線形近似した.
        近似式は\equref{length relation}となった.
        \begin{eqnarray}
          \equlabel{length relation}
          \lambda=-1.45l+0.907
        \end{eqnarray}
        \equref{lambda},\equref{length relation}をまとめると\equref{l status}となり,
        半周期後に現在振幅から目標振幅に到達するために必要な最大長は振幅から求めることができる.
        \begin{eqnarray}
          \equlabel{l status}
          l=\frac{1}{1.45}\left(0.907-\frac{\omega}{\pi}\ln{\left(\frac{A_{\mathrm{now}}}{A_{\mathrm{ref}}}\right)}\right)
        \end{eqnarray}
        \fig{photo.jpg}{width=1\hsize}{Relationship between maximum length and damping rate}
        

        \subsection{励振調整}
        
        \equlabel{l status}を用いて,現在の振幅と目標振幅の値を基に伸縮量制御し,励振調整を行う.
        ここで,現在振幅の値により必要な最大長がロボットの伸縮可能最大長0.74 mを超えた場合は0.74 mで励振を行い,
        次の半周期で目標振幅への到達を試みる.
        また,最大長をロボットの伸縮可能最小長0.56 mとすることにより,励振を含まない単純な減衰となる.

        まず,\secref{目標振幅の導出}で導出した目標振幅$A_{\mathrm{ref}}=89 deg$に基づき,励振調整実験を行った.
        ロボットの角度・角速度をIMUを用いてリアルタイムに計測し,角速度の正負が切り替わるタイミングの角度を現在振幅とした.
        ここで,現在振幅と目標振幅の許容誤差は3 degとした.
        実験結果を\figref{}に示す.
        \fig{photo.jpg}{width=1\hsize}{Excitation adjustment experiment}


        \section{目標振幅の導出}
          
          \seclabel{目標振幅の導出}
          まず,最適なバーリリース条件である角度$\varphi_{\mathrm{ref}}$・角速度$\dot{\varphi}_{\mathrm{ref}}$になるために必要な目標振幅を求める.
          \equref{Matrix}は減衰を考慮していない運動方程式であるため,
          伸縮せず,粘性減衰減衰を仮定すると
          次のように表される.ここで,$b$は粘性減衰係数を表す.
          \begin{eqnarray}
            \equlabel{dampingEquation}
            M_{11}(l)\ddot{\varphi}+b\dot{\varphi}+\frac{1}{2}mgl\sin{(\varphi)}=0          
            \end{eqnarray}
          実機を用いた減衰計測データに近い振動になるように減衰係数$b$を調整し,
          \equref{dampingEquation}を4次のルンゲクッタ法で解いた結果と減衰計測データを\figref{DampingData.eps}に示す.
          ここで,減衰係数は$b=0.050$ [Ns/m]とした.
          \fig{DampingData.eps}{width=1\hsize}{Damping experiment data and simulation}
          % \figref{}に減衰計測データ(角度・角速度)を示す.
          % なお,ロボットの長さを0. 74 m,初期角度を130 degとした.
          % \fig{photo3.jpg}{width=0.3\hsize}{Damped vibration}
          % この減衰データを基に最小二乗法を用いて指数近似を行うことで減衰係数を求めたところ$b=0.059$であった.
          % 求めた減衰係数を用いて\equref{dampingEquation}を4次のルンゲクッタ法で解いた結果を減衰計測データとともに
          % \figref{}に示す.全体的に振幅のずれがあるため,減衰係数を調整して$b=0.050$とした結果が\figref{}である.
          \figref{DampingData.eps}から,時間が経過するほど数値解と計測データとの間に振幅のずれが生じることが確認できた.
          このことから,実際の減衰では角速度比例だけではない減衰があると考えられる.
          しかし,本実験では目標振幅の導出には振動の半周期分かつ,主に45 deg以上の振幅のみを用いる.
          \figref{DampingData.eps}において時間経過後も振動数はほとんど一致しており,45 deg以上の範囲では振幅が一致していることから,適用可能であると考えた.
          調整した減衰係数を用いて\equref{dampingEquation}を初期角度を変えながら4次のルンゲクッタ法で解き,\chapref{chapter4}で求めた最適なバーリリース条件に最も近い目標振幅を求めた.
          なお,\equref{error}に示す誤差$e$が最小となる初期角度を目標振幅とした.
          \begin{eqnarray}
            \equlabel{error}
            e=\sqrt{(\varphi_{\mathrm{ref}}-\varphi)^2+(\dot{\varphi}_{\mathrm{ref}}-\dot{\varphi})^2}
            \end{eqnarray}
          これにより,\tabref{ExperimentConditions},\tabref{optimizedRelease}の条件では,
          目標振幅は$A_{\mathrm{ref}}=89\,\mathrm{deg}$と求まった.
          ここで,最適なバーリリース条件の角度は正であるため,目標振幅は振子の負の範囲における振幅である.

        





          



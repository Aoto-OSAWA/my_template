\include{my_layout_grad}


\usepackage{ikuo}%%便利コマンド集.



\usepackage[dvipdfmx]{hyperref}  % 目次や参考文献をリンクにする。
\usepackage{pxjahyper} %% これを入れるとしおりが文字化けしない。out2uniが不要になる。
%% \hypersetup{bookmarksnumbered=true}
\hypersetup{colorlinks=true}
\hypersetup{linkcolor=black}
%% \hypersetup{linkbordercolor=black}
\hypersetup{urlcolor=black}
%% \hypersetup{urlbordercolor=black}
\hypersetup{citecolor=black}
%% \hypersetup{citebordercolor=black}

\usepackage{url} % \url のために必要。パッケージが無い人は探して入れる。
%% \url{http://nile.ulis.ac.jp/~yuka/}のようにして使う。

\newcommand{\FIGDIR}{./fig}        %図を置くディレクトリを指定する

\hypersetup{
  pdfborder={0 0 0}   % リンクの枠線を無効化
}

\date{令和6年度卒業論文}
\title{伸縮する単リンクブラキエーションロボットの自在移動の実現}
\author{指導教員 水内 郁夫 教授 \\
\ \\
東京農工大学 \\
工学部 機械システム工学科 \\
\ \\
令和3年度入学\\
21265014\\
{\bf 大澤 蒼人}}

\begin{document}
\setlength{\baselineskip}{20pt}
\maketitle
\tableofcontents

%%各章は別ファイルにして以下にinculudeすると良い.
%卒論用
\chapter[序論]%
        {序論}
        \section{研究の背景と目的}
        \seclabel{1-1}

          ブラキエーションは,上肢で枝を掴んでぶら下がりながら移動する方法であり,重力を利用することで高所を効率的に移動できる.
          この移動方法をロボットに応用することで\cite{福田敏男1990ブラキエーション形移動ロボットの研究},送電線の点検などの高所作業への適用が期待される.
          テナガザルを模倣した多リンク型のロボットの研究例として,福田らの2リンク型\cite{福田敏男1991ブラキエーション形移動ロボットの研究2}\cite{福田敏男1992ブラキエーション形移動ロボットの研究}\cite{齋藤史倫1993ブラキエーション形移動ロボットの研究}\cite{齋藤史倫1995学習とロボット}\cite{福田敏男1996強化学習法を用いたファジィコントローラの生成}\cite{中西淳1998解析的手法による}\cite{中西淳19992}\cite{中西淳2001ハイブリッドコントローラによる}や
          5リンク型\cite{福田敏男1991ブラキエーション形移動ロボットの研究},6リンク型\cite{福田敏男1990ブラキエーション形移動ロボットの研究},7リンク型\cite{齋藤史倫1994ブラキエーション形移動ロボットの研究},13リンク型\cite{長谷川泰久2001ブラキエーション形移動ロボットの研究}などがある.
          また,把持機構に電磁石を用いた2リンク型\cite{山川雄司2016ブラキエーションロボットの開発と運動生成}\cite{山川雄司2016-2ブラキエーションロボットの開発と運動生成}や,
          パッシブグリッパーを用いた2リンク型\cite{javadi2023acromonk},3リンク型\cite{grama2024ricmonk}などがある.
          しかし,多リンク型は構造が複雑であるとともに,
          カオス現象\cite{鈴木三男2000二重振り子におけるカオス的振舞}が生じることで制御が難しくなるという問題がある.
          赤羽らはロボットの形状を棒状,すなわち単リンク型にすることで構造を単純化し,これらの問題を解決した\cite{akahane2022single}.
          また,おもりを


          異なる高さ、位置

          励振の調整は行っていなかった。

          さらに伸縮

          空中過程(空中相にしないように)

          本研究では,バーの位置に基づいた最適なバーリリース条件を導出し,その条件による空中過程を含む移動により,
          伸縮する単リンクブラキエーションロボットの自在移動を実現することを目的とする.
          伸縮する機構を活かした最適なバーリリース条件の導出と励振制御を




          実験的に得た時刻を基に再計画は行っているが、相対速度を考慮していないためロバストではない

          伸縮することでバーの位置によってリリース時の長さを変え、

          時刻ではなく角度角速度にすることで、リアルタイムに計測していることにより励振プログラムが実行された後に不具合が生じてその時刻に適切な状態になくても

          空中過程(跳躍 飛ぶ動作 次のバーを掴む前に支持していたグリッパーもバーから離す)跳躍ブラキエーション
          通常のブラキエーションよりも高速かつ遠くの目標物まで到達可能
          
          バーとの相対速度が大きいことで、衝突により把持するタイミングがずれることや部品破損といったことが生じる可能性がある。

          伸縮調整により、以前はその時間になるまで待っていたけどより速く到達できる(早くの評価はいまいちかも)


        \section{本論文の構成}
        \seclabel{structure}

          本論文は,全XXX章から構成させる.以下に各章の概要を述べる.
          \begin{itemize}
            \item 第1章(本章)では,研究の背景と目的について述べた.
            \item 第2章「本研究」では
            \item 第3章「最適なバーリリース条件の導出」ではバーの位置に基づく最適なバーリリース条件を導出する.
            \item 第4章「励振制御」ではXXX.
            \item 第5章「空中過程を含むブラキエーション実験」ではXXX.
            \item 第6章「結論および今後の展望」ではXXX.
          \end{itemize}


\chapter[本研究におけるブラキエーション動作と実機構成]%
        {本研究におけるブラキエーション動作と\\実機構成}
        \section{はじめに}
          
          本章では,目的とするブラキエーション動作と伸縮による励振,そして本研究で用いるロボットの
          実機構成について述べる.


        \section{ブラキエーションの流れ}

          \figref{brachiationFig-4.eps}に本研究で目的とする空中過程を含んだブラキエーション動作を示す.
          ロボットの両端のグリッパーがバーを掴んだ状態(\figref{brachiationFig-4.eps}(1))から,
          片方のバーを離して振子過程(\figref{brachiationFig-4.eps}(2))に移る.
          目標のバーまでの距離がロボットの最大長以下である場合,バーの距離に合わせてロボットを伸縮させることで
          空中過程を含まないブラキエーション(\figref{brachiationFig-4.eps}(3-1))を行う.
          一方,目標のバーまでの距離がロボットの最大長以上である場合,適切なタイミングでもう一方のバーも離し(\figref{brachiationFig-4.eps}(3-2)),
          空中過程を経て,最後に目標のバーを掴む(\figref{brachiationFig-4.eps}(4)).
          これらの動きを繰り返すことで連続したブラキエーションを行う.

        \section{伸縮による励振}
          
          目標とするバーの位置が,把持していたバー(\figref{brachiationFig-4.eps}におけるbar1)と同じ,もしくはそれよりも高い場合,
          振子過程においてロボットの振幅を大きくしなければ空気抵抗や摩擦などの影響により目標のバーへ到達することができない.
          そのため,振子過程において外部からのエネルギー投入による振動の拡大が望まれる.
          本研究ではロボットを伸縮することにより,ブランコのように重心位置をロボットの長手方向に変化させることで励振を行う.
          この励振方法は先行研究があり,実機で実現されている\cite{Hijiri:Robomech2024}.
          
          \figb{brachiationFig-4.eps}{width=0.9\hsize}{Brachiation motions}
        
          % \newpage
        \section{伸縮する単リンクブラキエーションロボットの実機構成}
          
          本研究では\cite{Hijiri:Robomech2024}で使用していた実機を用いた.
          \figref{bar-robot4.png}に全体図を示す.重さは3.0 kgであり,幅・奥行・高さは
          最も縮めた場合は$150{\times}80{\times}560$ mm,最も伸ばした場合は$150{\times}80{\times}740$ mmである.
          伸縮にはラックピニオン機構が用いられており,中心部のブラシレスDCモータ(MAXON EC22 40W, ギアヘッドギア比128)によって歯車を回転させる.
          ブラシレスDCモータはモータドライバ(EPOS2 24/5)を介してロボット全体の制御を行うArduino Unoに接続されている.
          また,グリッパーは把持部品がサーボモジュール(双葉電子工業 RS406CB)で駆動する.
          ロボット全体の姿勢取得にはIMU(Adafruit BNO055)を用いている.
          
          \fig{photo4.jpg}{width=0.5\hsize}{Overall view of extensible brachiation robot}

            
        % \section{自在移動のためのリリース条件と励振制御}
        %   空中過程を含むブラキエーション動作の場合,
        %   落下せずに移動するためにはより正確かつ把持時の衝撃の小さいバー把持が望まれる.
        %   バーの位置が異なる場合でも,バーの位置に基づいてロボットの長さ,リリース条件を求めることによって
        %   より理想的なブラキエーションを
          
        %   次のバーの位置に基づいて伸縮,そしてバーリリースを行うことで最適な
        %   単リンク構造は振子過程・空中過程の両方においてロボットの動きを制御しやすい.
        %   そのため,ロボットの手先軌道を求めやすく,最適なバーリリース条件を決めやすい.
        %   実験的にバーを離すタイミングを決めることもできるが、
        %   バーの座標が変わった時に
        %   実用的ではない
\chapter[バーの位置に基づく空中過程を含まないブラキエーション動作]%
{バーの位置に基づく\\空中過程を含まないブラキエーション動作}
        \chaplabel{chapter3}
        \section{はじめに}
        
        伸縮する棒状ブラキエーションロボットは,その「伸縮する」という特徴を活用することで,
        目標のバーまでの距離に合わせてリンクの長さを調整して移動することが可能となる.
        これにより,おもりを動かすことで励振するブラキエーションロボット\cite{akahane2022single}よりも
        自在な移動が実現できる.
        先行研究\cite{Hijiri:Robomech2024}では実験的にリンクを伸ばすタイミングを決定していた.
        しかし,伸ばすタイミングや長さが適切ではない場合,バーを把持できないだけでなく,
        バーとロボットが衝突することで不具合が生じたり,破損してしまう.
        また,目標バーを異なる位置にした場合,再び実験的にリンクを伸ばすタイミングを模索しなければならない.
        そこで,本研究ではバーの位置に基づいて振幅を調整し,リンクを伸ばすタイミングを決定し,空中過程を含まないブラキエーション動作を行った.
          
        \section{伸縮タイミングの検討}

        リンクを伸ばして次のバーを把持する場合,伸ばすタイミングは\figref{noAerialBrachiationFig.eps}に示すように以下の4通りの方法が考えられる.
        なお\figref{gripper.eps}に示すように,グリッパーは開いた状態では爪がリンクの先端からはみ出さない構造になっているため,メインボディがbar1やbar3に接触しない場合は
        グリッパーも接触することなく通過することが可能である.

        \begin{itemize}
        \item 方法1 (\figref{noAerialBrachiationFig.eps}(1)):
        
        バーに対して下側から近づき,バーに近づいたらリンクを伸ばし始める.

        バー把持時のグリッパーに対する回転方向の負荷が小さいが,
        伸縮による角速度の変化や,伸縮時間を考慮する必要があり伸縮量操作が複雑になる.
        
        \item 方法2 (\figref{noAerialBrachiationFig.eps}(2)):
        
        バーに対して上側から近づき,バーに近づいたらリンクを伸ばし始める.
        
        バー把持時のグリッパーに対する回転方向の負荷が小さいが,
        伸縮による角速度の変化や,伸縮時間を考慮する必要があり伸縮量操作が複雑になる.

        \item 方法3 (\figref{noAerialBrachiationFig.eps}(3)):
        
        最高点においてリンクを目標の長さまで伸ばし終えた状態で上側からバーに近づく.
        
        振子の最高点に達したときに伸び始めるため,制御が容易であるが,
        最高点が目標のバーから離れている場合,バー把持時のグリッパーに対する回転方向の負荷が大きくなる.
        
        \item 方法4 (\figref{noAerialBrachiationFig.eps}(4)):
        
        最高点においてリンクを目標の長さまで伸ばし終えた状態で下側からバーに近づく.

        方法3と同様に伸縮量操作が容易であるが,
        バー把持時のグリッパーに対する回転方向の負荷が大きくなる.
        また,把持していたバー(\figref{noAerialBrachiationFig.eps}におけるbar 1)と衝突する可能性がある.

        \end{itemize}

        ここで,振動の最高点においてリンクを伸ばし始めた場合を考えた.
        ロボットが最小長0.56 mから最大長0.74 mまで伸びるためにかかる時間を計測したところ
        0.37 sであり,\equref{Approximation Model}を用いて伸び終えるまでの角度変化量を求めたところ,
        振幅によって少なくとも約40 deg,多い場合だと約70 degであることが分かった.
        すなわち,方法2・方法3では伸び終える前に目標バーを通り過ぎてしまう可能性がある.
        そこで,本研究では伸縮のための時間が十分あり,伸縮量操作が容易である方法4を採用し,bar1を通り過ぎた後に目標長さに変えることでバーとの衝突を回避することを想定した.
        \fig{noAerialBrachiationFig.eps}{width=1.0\hsize}{Approach methods in brachiation movements without an aerial phase}

        \section{空中過程を含まないブラキエーション実験}

        \subsection{目標振幅の決定}
        \seclabel{mokuhyou}
        \figref{BarPositionFig.eps}に目標とするバーの位置の概略図を示す.
        座標軸は左向きを$x$軸の正方向,上向きを$z$軸の正方向に設定し,
        ロボットが把持しているバーの座標を原点$(0,0)$,
        目標バーの座標を$(l_{\mathrm{bx}},l_{\mathrm{bz}})$とし,原点と結んだ線分が$z$軸となす角度を$\theta$とする.
        また,ロボット姿勢は角度$\varphi$,角速度$\dot{\varphi}$とボディの全長$l_{\mathrm{r}}$で表す.
        グリッパーが届く範囲にあるバーであれば,空中過程を含まないブラキエーションが可能である.
        ゆえに,ロボットの最大の長さを$l_{\mathrm{rMax}}$,最小の長さを$l_{\mathrm{rMin}}$とすると
        目標とするバーの条件は\equref{barRange}となる.
        また,目標とするバーの座標を用いて角度$\theta$は\equref{theta}で表される.
        \begin{eqnarray}
                \equlabel{barRange}
                l_{\mathrm{rMin}} &\le& \sqrt{l_{\mathrm{bx}}^2+l_{\mathrm{bz}}^2} \le l_{\mathrm{rMax}}\\
                \equlabel{theta}
                \theta&=&90+\arctan{\left(\frac{l_{\mathrm{bz}}}{l_{\mathrm{bx}}}\right)}
                \end{eqnarray}  
        ここで,グリッパーを閉じる時間を考慮して目標バーの高さに到達する前にグリッパーを閉じ始めると,
        グリッパーの上側の爪(\figref{BarPositionFig.eps}におけるclaw1)がバーに衝突してしまう.そこで,目標バーの高さを超えてから閉じ始め,
        下側の爪(\figref{BarPositionFig.eps}におけるclaw2)が引っかからない状態で目標バーを通り過ぎ,
        目標バーの高さになったタイミングでグリッパーが目標バーに引っかかる状態まで閉じているようにした.
        \figref{noAerialAmp.eps}に示す振幅を用いてバー把持の流れを以下にようにした.
        \begin{enumerate}
                \item 振幅$A_0$になるように振幅調整
                \item 目標リンク長(目標バーまでの距離)に伸縮調整
                \item 角速度の正負が入れ替わるタイミング(振幅$A_1$)でグリッパーを閉じ始める
                \item 目標バーの位置$\theta$でグリッパーが目標バーに引っかかる
                \item グリッパーを閉じ終える
        \end{enumerate}      
        グリッパーを閉じる時間と\equref{Approximation Model}を基に目標振幅となる$A_0$を求めた.
        振幅$A_0$に到達した後は単純な減衰振動となると仮定した.
        ここで,異なるロボット長$l_{\mathrm{r}}$で単純な減衰振動データを計測し,それぞれ\figref{damping.eps}に示すように\equref{edamp}の形で指数関数近似することで,
        振幅$A$の減衰率$\gamma$を求めた.いずれの長さでも減衰率は近い値であった\tabref{damprate}にそれぞれの減衰率とその平均を示す.
        さらに,その平均値は\equref{length relation}において単純な減衰運動となる$l_{\mathrm{max}} =$ 0.56 mとした時の振幅増加率$\lambda =$ -0.04とほとんど同じであったため,
        本研究のロボット長の変化範囲では減衰率は長さに依存せずに一定値として扱った.
        よって$A_0$から$A_1$までの単純な減衰振動は,\equref{Approximation Model}において振幅増加率$\lambda=-0.04$として考えた.
        グリッパーを閉じるために要する時間が0.36 s であり,バーに対して引っかかることができる状態(グリッパーモジュール回転量50 deg)までは$t_{\mathrm{gripper}} =$0.18 sであることから,閉じ終わりが目標バー位置$\theta$であるために必要な振幅$A_1$は\equref{A1}となる.
        また,半周期後に振幅$A_1$であるために必要な振幅$A_0$は\equref{A0}となる.
        なお,$\omega_{(A)}$は振幅$A$を基に求めた角振動数を表す.
        これにより,振幅$A_0$を空中過程を含まないブラキエーションのための目標振幅とした.
        \begin{eqnarray}
                \equlabel{edamp}
                A(t)=A e^{\gamma t}
        \end{eqnarray}
        \begin{eqnarray}
                \equlabel{A1}
                A_1&=&\theta e^{-(\lambda)\times t_{\mathrm{gripper}}}\cos{(\omega_{(A_1)}\times t_{\mathrm{gripper}})}\\
                \equlabel{A0}
                A_0&=&A_1 e^{-(\lambda)\times\pi/\omega_{(A_0)}}\cos{(\omega_{(A_0)}\times\pi/\omega_{(A_0)})}
                \end{eqnarray}
        \begin{table}[bh]
                \begin{center}
                        \caption{Damping rate data}
                        \tablabel{damprate}
                        \vspace{2mm}
                        \begin{tabular}{c|c}
                        \hline
                        $l_{\mathrm{r}}$ [m] & $\gamma$ [1/s]\\
                        \hline
                        0.56 & -0.0386\\ 
                        0.62 & -0.0471\\ 
                        0.68 & -0.0393\\
                        0.74 & -0.0356\\                         
                        \hline
                        \hline
                        Average & -0.0402\\                      
                        \hline
                        \end{tabular}
                \end{center}
                \end{table}
        \fig{BarPositionFig.eps}{width=0.7\hsize}{Schematic diagram of bar position}
        \fig{noAerialAmp.eps}{width=0.8\hsize}{Schematic diagram of amplitude}
        \fig{damping.eps}{width=1.0\hsize}{Schematic diagram of amplitude}
        
        \newpage
        \subsection{ブラキエーション実験}

        空中過程を含まないブラキエーション動作の実験として,もともと把持しているバーと
        同じ高さ(実験1)・異なる高さ(実験2)の2種類の位置に目標バーを設置して実験を各10回行った.
        目標バーの位置$(l_{\mathrm{bx}},l_{\mathrm{bz}})$,$\theta$,振幅$A_1$,目標振幅$A_0$を
        それぞれ\tabref{ex1}\tabref{ex2}に示す.
        \begin{table}[t]
                \begin{minipage}[c]{0.5\hsize}
                  \centering
                  \caption{No aerial phase experiment 1}
                  \tablabel{ex1}
                  \vspace{2mm}
                  \begin{tabular}{c|c}
                    \hline
                    $l_{\mathrm{bx}}$ & 0.74 m \\
                    $l_{\mathrm{bz}}$ & 0.00 m \\
                    $\theta$ & 90 deg \\ 
                    $A_1$ & 108 deg \\
                    $A_0$ & 113 deg \\
                    \hline
                  \end{tabular}
                \end{minipage}
                \begin{minipage}[c]{0.5\hsize}
                  \centering
                  \caption{No aerial phase experiment 2}
                  \tablabel{ex2}
                  \vspace{2mm}
                  \begin{tabular}{c|c}
                    \hline
                    $l_{\mathrm{bx}}$ & 0.72 m \\
                    $l_{\mathrm{bz}}$ & 0.14 m \\
                    $\theta$ & 101 deg \\ 
                    $A_1$ & 120 deg \\
                    $A_0$ & 126 deg \\
                    \hline
                  \end{tabular}
                \end{minipage}
              \end{table}
        
        また,各10回の計測のうち,それぞれ1つずつ成功データを抜粋し,計測データ(角度)とロボット最大長指令値の時間変化を\figref{NoAerial1data.eps}と\figref{NoAerial2data.eps},
        実験の様子を\figref{NoAerial.eps}に示す.なお,実験の様子は\figref{NoAerial1data.eps},\figref{NoAerial2data.eps}において
        赤線で示した時刻からの様子である.振幅$A_1$,目標振幅$A_0$は実験結果では\tabref{A0A1}に示すように
        それぞれ$A_{\mathrm{1Real}}$,$A_{\mathrm{0Real}}$となった.なお,\tabref{ex1}\tabref{ex2}に示した振幅との誤差を括弧の数字で表す.
        目標振幅の誤差は実験1,実験2ともに小さく,提案した振幅調整法の有効性を確認できた.
        一方で,振幅$A_1$との誤差はかなり大きく,バーとの摩擦により想定よりも減衰したと考えられる.
        
        ブラキエーション実験の成功率としては実験1,実験2のどちらの条件においても60%であった.
        実験1の条件における失敗時の角度データと振幅$A_1$・目標振幅$A_0$の実験結果をそれぞれ\figref{NoAerialFailed.eps},\tabref{failed}に示す.
        また,10回の計測のうち,成功時・失敗時のそれぞれの振幅$A_1$・目標振幅$A_0$の平均,標準偏差を\tabref{dataanalysis}に示す.
        \tabref{dataanalysis}より,成功時・失敗時共に振幅$A_1$の標準偏差が大きく,ばらつきがあることが分かった.
        一方で,平均値に関しては,目標振幅$A_0$は成功時・失敗時共に近い値となったが,振幅$A_1$は成功時と失敗時の値の差が大きく,失敗時は小さいことが分かった.
        このことから,失敗の要因として実験ごとに目標振幅$A_0$から振幅$A_1$に到達するまでの間の振幅減衰量に影響を無視できないほどのばらつきがあると考えられる.
        また,\tabref{failed}より,振幅$A_{\mathrm{1Real}}$が目標バーの位置$\theta$よりも小さいことが確認でき,グリッパーが閉じ始めた時には目標バーを下回っていたとみられる.
        その原因として,目標バーとの接触が挙げられる.\figref{NoAerialFailed.eps}において目標バーと衝突した時刻を赤線で示した.
        グリッパーの構造上,振子の回転軸方向に振動が発生しているため,
        \figref{bar-gripper.jpg}に示すように,グリッパーの歯車部分が目標バーと接触してしまう状態であったと考えられる.
        \begin{table}[bh]
                \begin{center}
                  \caption{No aerial phase experiment amplitude data}
                  \tablabel{A0A1}
                  \vspace{2mm}
                  \begin{tabular}{c|cc}
                    \hline
                     & Experiment 1 & Experiment 2\\
                    \hline
                    $A_{\mathrm{1Real}}$ & 94.1 (12.9%)& 100.8 (16%)\\
                    $A_{\mathrm{0Real}}$ & 112.9 (0.0%)& 128.7 (2.1%)\\                      
                    \hline
                  \end{tabular}
                \end{center}
              \end{table}
              \begin{table}[bh]
                \begin{center}
                  \caption{Measurement data(Experiment 1)}
                  \tablabel{dataanalysis}
                  \vspace{2mm}
                  \begin{tabular}{c|cc}
                    \hline
                     & Average & Standard deviation\\
                    \hline
                    $A_0$(Succeeded)& 113.72& 1.59\\
                    $A_1$(Succeeded)& 100.59& 5.65\\
                    $A_0$(Failed)& 113.87& 0.31\\
                    $A_1$(Failed)& 94.30& 3.03\\                      
                    \hline
                  \end{tabular}
                \end{center}
              \end{table}
        \newpage
        \fig{NoAerial1data.eps}{width=0.6\hsize}{No aerial phase experiment 1}
        \fig{NoAerial2data.eps}{width=0.6\hsize}{No aerial phase experiment 2}
        \fig{NoAerial.eps}{width=0.65\hsize}{No aerial phase experiment flow}
        \clearpage
        \begin{table}[t]
                \begin{center}
                  \caption{No aerial phase failed experiment amplitude data}
                  \tablabel{failed}
                  \vspace{2mm}
                  \begin{tabular}{c|c}
                    \hline
                    $A_{\mathrm{1Real}}$ & 87.1 (19.4%)\\
                    $A_{\mathrm{0Real}}$ & 114.6 (1.4%)\\                      
                    \hline
                  \end{tabular}
                \end{center}
              \end{table}
        \fig{NoAerialFailed.eps}{width=0.6\hsize}{No aerial phase failed experiment data}
        \fig{bar-gripper.jpg}{width=0.5\hsize}{Contact with the bar}
        
              
\chapter[リリース条件到達のための励振制御]%
{リリース条件到達のための励振制御}
        \section{はじめに}
        \seclabel{4-1}

          \chapref{chapter3}において導出した最適なバーリリース条件を,振子過程において実現させる.


        \section{伸縮による励振}
        \section{振子過程での運動方程式}

\chapter[リリース条件到達のための励振制御]%
{リリース条件到達のための励振制御}
        \section{はじめに}

        \chapref{chapter4}において導出した最適なバーリリース条件を得るためには,
        振子過程における励振調整が望まれる.
        本章では伸縮タイミングと伸縮量の調整による目標角度・角速度の実現について述べる.
          
        \section{伸縮による励振}

          本研究では,重心を移動させることにより振子過程において励振させる.
          Lieskovsk{\`y}らの重りを動かすことによる振幅増幅率が最大となる重心移動の研究\cite{lieskovsky2023optimal}を,
          伸縮機構に応用して励振を行う.\figref{}に伸縮する単リンクブラキエーションロボットを簡略化した,
          伸縮する振子のモデルを示す.振子の半径方向と鉛直下向き線がなす角を$\varphi [\mathrm{rad}]$,
          振子の長さを$l [\mathrm{m}]$,振子の質量を$m [\mathrm{kg}]$,振子の重心周りの慣性モーメントを$I [\mathrm{kg}\mathrm{m}^2]$,
          重力加速度を$g [\mathrm{m}/\mathrm{s}^2]$とする.運動方程式を以下のラグランジュの運動方程式で求める.

          \begin{eqnarray}
            \equlabel{Lagrange}
            \frac{\mathrm{d}}{\mathrm{d}t}\left(\frac{\partial L}{\partial \dot{q_i}}\right)-\left(\frac{\partial L}{\partial q_i}\right)=S_i
            \end{eqnarray} 
          
          $q_i$と$S_i$はそれぞれ一般化座標と一般化力であり,それぞれ\equref{q}\equref{S}とした.
          $L=T-U$はラグランジアンであり,
          系の運動エネルギー$T$と位置エネルギー$U$で構成され,本モデルでは\equref{Tequ}\equref{Uequ}となる.  
          \begin{eqnarray}
              \equlabel{qequ}
              \begin{bmatrix}
                q_1 \\
                q_2 \\
                \end{bmatrix}
              =
              \begin{bmatrix}
                \varphi \\
                l \\
                \end{bmatrix}\\

              \equlabel{Sequ}
              \\
              \equlabel{Tequ}
              T = \frac{1}{2} \\
              \equlabel{Uequ}
              U = -gm\frac{l}{2}\cos(\varphi)
              \end{eqnarray}  
          運動方程式を行列で書き表すと\equref{Matrix}となる.
          \begin{eqnarray}
            \equlabel{Matrix}
            \left[\right]
            \end{eqnarray} 
          ここで,XXXXXXXXは以下のようになる.
            \begin{eqnarray}
              \equlabel{M}
              x_{\mathrm{c}}&=&\frac{1}{2}l_{\mathrm{r}}\dot{\varphi}\cos{(\varphi)}t+\frac{1}{2}l_{\mathrm{r}}\sin{(\varphi)}\\
              \equlabel{C}
              z_{\mathrm{c}}&=&\frac{1}{2}l_{\mathrm{r}}\dot{\varphi}\sin{(\varphi)}t-\frac{1}{2}gt^2-\frac{1}{2}l_{\mathrm{r}}\cos{(\varphi)}\\
              \equlabel{M11}
              x_{\mathrm{e}}&=&x_{\mathrm{c}}+\frac{1}{2}l_{\mathrm{r}}\sin{(\varphi+\dot{\varphi}t)}\\
              \equlabel{M22}
              z_{\mathrm{e}}&=&z_{\mathrm{c}}-\frac{1}{2}l_{\mathrm{r}}\cos{(\varphi+\dot{\varphi}t)}\\
              \equlabel{C1}
                J_{\mathrm{d}}(\varphi,\dot{\varphi},t,l_{\mathrm{r}})
                &=&\sqrt{(l_{\mathrm{bx}}-x_{\mathrm{e}})^2+(l_{\mathrm{bz}}-z_{\mathrm{e}})^2}\\
              \equlabel{C2}
              J_{\mathrm{r}}(\varphi,\dot{\varphi},t,l_{\mathrm{r}})
              &=&\sqrt{\dot{x_{\mathrm{e}}}^2+\dot{z_{\mathrm{e}}}^2}
            \end{eqnarray} 





            \begin{eqnarray}
              \equlabel{lr-range}
              x_{\mathrm{c}}&=&\frac{1}{2}l_{\mathrm{r}}\dot{\varphi}\cos{(\varphi)}t+\frac{1}{2}l_{\mathrm{r}}\sin{(\varphi)}\\
              \equlabel{bar}
              z_{\mathrm{c}}&=&\frac{1}{2}l_{\mathrm{r}}\dot{\varphi}\sin{(\varphi)}t-\frac{1}{2}gt^2-\frac{1}{2}l_{\mathrm{r}}\cos{(\varphi)}\\
              \equlabel{x-e}
              x_{\mathrm{e}}&=&x_{\mathrm{c}}+\frac{1}{2}l_{\mathrm{r}}\sin{(\varphi+\dot{\varphi}t)}\\
              \equlabel{z-e}
              z_{\mathrm{e}}&=&z_{\mathrm{c}}-\frac{1}{2}l_{\mathrm{r}}\cos{(\varphi+\dot{\varphi}t)}\\
              \equlabel{Jd}
                J_{\mathrm{d}}(\varphi,\dot{\varphi},t,l_{\mathrm{r}})
                &=&\sqrt{(l_{\mathrm{bx}}-x_{\mathrm{e}})^2+(l_{\mathrm{bz}}-z_{\mathrm{e}})^2}\\
              \equlabel{Jr}
              J_{\mathrm{r}}(\varphi,\dot{\varphi},t,l_{\mathrm{r}})
              &=&\sqrt{\dot{x_{\mathrm{e}}}^2+\dot{z_{\mathrm{e}}}^2}
            \end{eqnarray} 
        \section{振子過程での運動方程式}
        振子過程のシミュレーション説明
        2つの過程で調整を行う
          減衰
            数値解  
            解析解
            
            減衰データとフィッティング
          励振

          調整プログラム



\chapter[空中過程を含むブラキエーション実験]%
{空中過程を含むブラキエーション実験}
        \section{はじめに}
        \seclabel{5-1}

        \fig{photo.jpg}{width=1\hsize}{Excitation experiment data and fitting data}
        \section{目標振幅の導出}
          
        \seclabel{目標振幅の導出}
        まず,最適なバーリリース条件である角度$\varphi_{\mathrm{ref}}$・角速度$\dot{\varphi}_{\mathrm{ref}}$になるために必要な目標振幅を求める.
        \equref{Matrix}は減衰を考慮していない運動方程式であるため,
        伸縮せず,粘性減衰減衰を仮定すると
        次のように表される.ここで,$b$は粘性減衰係数を表す.
        \begin{eqnarray}
          \equlabel{dampingEquation}
          M_{11}(l)\ddot{\varphi}+b\dot{\varphi}+\frac{1}{2}mgl\sin{(\varphi)}=0          
          \end{eqnarray}
        実機を用いた減衰計測データに近い振動になるように減衰係数$b$を調整し,
        \equref{dampingEquation}を4次のルンゲクッタ法で解いた結果と減衰計測データを\figref{DampingData.eps}に示す.
        ここで,減衰係数は$b=0.050$ [Ns/m]とした.
        \fig{DampingData.eps}{width=1\hsize}{Damping experiment data and simulation}
        % \figref{}に減衰計測データ(角度・角速度)を示す.
        % なお,ロボットの長さを0. 74 m,初期角度を130 degとした.
        % \fig{photo3.jpg}{width=0.3\hsize}{Damped vibration}
        % この減衰データを基に最小二乗法を用いて指数近似を行うことで減衰係数を求めたところ$b=0.059$であった.
        % 求めた減衰係数を用いて\equref{dampingEquation}を4次のルンゲクッタ法で解いた結果を減衰計測データとともに
        % \figref{}に示す.全体的に振幅のずれがあるため,減衰係数を調整して$b=0.050$とした結果が\figref{}である.
        \figref{DampingData.eps}から,時間が経過するほど数値解と計測データとの間に振幅のずれが生じることが確認できた.
        このことから,実際の減衰では角速度比例だけではない減衰があると考えられる.
        しかし,本実験では目標振幅の導出には振動の半周期分かつ,主に45 deg以上の振幅のみを用いる.
        \figref{DampingData.eps}において時間経過後も振動数はほとんど一致しており,45 deg以上の範囲では振幅が一致していることから,適用可能であると考えた.
        調整した減衰係数を用いて\equref{dampingEquation}を初期角度を変えながら4次のルンゲクッタ法で解き,\chapref{chapter4}で求めた最適なバーリリース条件に最も近い目標振幅を求めた.
        なお,\equref{error}に示す誤差$e$が最小となる初期角度を目標振幅とした.
        \begin{eqnarray}
          \equlabel{error}
          e=\sqrt{(\varphi_{\mathrm{ref}}-\varphi)^2+(\dot{\varphi}_{\mathrm{ref}}-\dot{\varphi})^2}
          \end{eqnarray}
        これにより,\tabref{ExperimentConditions},\tabref{optimizedRelease}の条件では,
        目標振幅は$A_{\mathrm{ref}}=89\,\mathrm{deg}$と求まった.
        ここで,最適なバーリリース条件の角度は正であるため,目標振幅は振子角度の負の範囲における振幅である.

      

        \section{考察}

\include{introduction}
\include{fig_tab_equ}

\addcontentsline{toc}{chapter}{謝辞}
\markboth{謝辞}{謝辞}
\chapter*{謝辞}
卒業論文を執筆するに当たり,水内郁夫教授より多大なるご指導,ご鞭撻を賜りました.
多くの技術,知識をこの一年間で学ばせていただきました.深く感謝申し上げます.
また,森下克幸助教にも論文執筆や発表技術に関するアドバイスをいただき,
Professor Tom{\'a}{\v{s} Vyhl{\'\i}dal,Mr. Juraj Lieskovsk{\`y}をはじめ,
Czech Technical University in Pragueの皆様には
留学時に多くのアドバイスをいただき,英語が不安な私をサポートしてくださいました.
感謝申し上げます.
さらに,研究室の先輩,同期には研究の面,そして研究以外の面でも支えていただきました.
本当にありがとうございました.


% \begin{verbatim}
% %% \addcontentsline{toc}{chapter}{謝辞}
% %% \markboth{謝辞}{謝辞}
% %% \chapter*{謝辞}
卒業論文を執筆するに当たり,水内郁夫教授より多大なるご指導,ご鞭撻を賜りました.
多くの技術,知識をこの一年間で学ばせていただきました.深く感謝申し上げます.
また,森下克幸助教にも論文執筆や発表技術に関するアドバイスをいただき,
Professor Tom{\'a}{\v{s} Vyhl{\'\i}dal,Mr. Juraj Lieskovsk{\`y}をはじめ,
Czech Technical University in Pragueの皆様には
留学時に多くのアドバイスをいただき,英語が不安な私をサポートしてくださいました.
感謝申し上げます.
さらに,研究室の先輩,同期には研究の面,そして研究以外の面でも支えていただきました.
本当にありがとうございました.


% \begin{verbatim}
% %% \addcontentsline{toc}{chapter}{謝辞}
% %% \markboth{謝辞}{謝辞}
% %% \chapter*{謝辞}
卒業論文を執筆するに当たり,水内郁夫教授より多大なるご指導,ご鞭撻を賜りました.
多くの技術,知識をこの一年間で学ばせていただきました.深く感謝申し上げます.
また,森下克幸助教にも論文執筆や発表技術に関するアドバイスをいただき,
Professor Tom{\'a}{\v{s} Vyhl{\'\i}dal,Mr. Juraj Lieskovsk{\`y}をはじめ,
Czech Technical University in Pragueの皆様には
留学時に多くのアドバイスをいただき,英語が不安な私をサポートしてくださいました.
感謝申し上げます.
さらに,研究室の先輩,同期には研究の面,そして研究以外の面でも支えていただきました.
本当にありがとうございました.


% \begin{verbatim}
% %% \addcontentsline{toc}{chapter}{謝辞}
% %% \markboth{謝辞}{謝辞}
% %% \include{thanks}
% \end{verbatim}

% emacs の人は、M-x comment-region ですね。
% コメント解除は、C-u M-x comment-region ですね。


% \end{verbatim}

% emacs の人は、M-x comment-region ですね。
% コメント解除は、C-u M-x comment-region ですね。


% \end{verbatim}

% emacs の人は、M-x comment-region ですね。
% コメント解除は、C-u M-x comment-region ですね。



\addcontentsline{toc}{chapter}{参考文献}
\markboth{参考文献}{参考文献}
\bibliographystyle{junsrt}
\bibliography{reference}

\end{document}

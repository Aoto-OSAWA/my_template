\chapter[序論]%
        {序論}
        \section{研究の背景と目的}
        \seclabel{1-1}

          ブラキエーションは,上肢で枝を掴んでぶら下がりながら移動する方法であり,重力を利用することで高所を効率的に移動できる.
          この移動方法をロボットに応用することで\cite{福田敏男1990ブラキエーション形移動ロボットの研究},送電線の点検などの高所作業への適用が期待される.
          テナガザルを模倣した多リンク型のロボットの研究例として,福田らの2リンク型\cite{福田敏男1991ブラキエーション形移動ロボットの研究2}\cite{福田敏男1992ブラキエーション形移動ロボットの研究}\cite{齋藤史倫1993ブラキエーション形移動ロボットの研究}\cite{齋藤史倫1995学習とロボット}\cite{福田敏男1996強化学習法を用いたファジィコントローラの生成}\cite{中西淳1998解析的手法による}\cite{中西淳19992}\cite{中西淳2001ハイブリッドコントローラによる}や
          5リンク型\cite{福田敏男1991ブラキエーション形移動ロボットの研究},6リンク型\cite{福田敏男1990ブラキエーション形移動ロボットの研究},7リンク型\cite{齋藤史倫1994ブラキエーション形移動ロボットの研究},13リンク型\cite{長谷川泰久2001ブラキエーション形移動ロボットの研究}などがある.
          また,把持機構に電磁石を用いた2リンク型\cite{山川雄司2016ブラキエーションロボットの開発と運動生成}\cite{山川雄司2016-2ブラキエーションロボットの開発と運動生成}や,
          パッシブグリッパーを用いた2リンク型\cite{javadi2023acromonk},3リンク型\cite{grama2024ricmonk}などがある.
          しかし,多リンク型は構造が複雑であるとともに,
          カオス現象\cite{鈴木三男2000二重振り子におけるカオス的振舞}が生じることで制御が難しくなるという問題がある.
          赤羽らはロボットの形状を棒状,すなわち単リンク型にすることで構造を単純化し,これらの問題を解決した\cite{akahane2022single}.
          また,おもりを動かす\cite{lieskovsky2023optimal}、伸縮することで\cite{Hijiri:Robomech2024}
          モデル予測制御\cite{Hijiri:Robomech2024-1}

          異なる高さ、位置

          励振の調整は行っていなかった。

          さらに伸縮

          空中過程(空中相にしないように)

          本研究では,バーの位置に基づいた最適なバーリリース条件を導出し,その条件による空中過程を含む移動により,
          伸縮する単リンクブラキエーションロボットの自在移動を実現することを目的とする.
          伸縮する機構を活かした最適なバーリリース条件の導出と励振制御を




          実験的に得た時刻を基に再計画は行っているが、相対速度を考慮していないためロバストではない

          伸縮することでバーの位置によってリリース時の長さを変え、

          時刻ではなく角度角速度にすることで、リアルタイムに計測していることにより励振プログラムが実行された後に不具合が生じてその時刻に適切な状態になくても

          空中過程(跳躍 飛ぶ動作 次のバーを掴む前に支持していたグリッパーもバーから離す)跳躍ブラキエーション
          通常のブラキエーションよりも高速かつ遠くの目標物まで到達可能
          
          バーとの相対速度が大きいことで、衝突により把持するタイミングがずれることや部品破損といったことが生じる可能性がある。

          伸縮調整により、以前はその時間になるまで待っていたけどより速く到達できる(早くの評価はいまいちかも)


        \section{本論文の構成}
        \seclabel{structure}

          本論文は,全XXX章から構成させる.以下に各章の概要を述べる.
          \begin{itemize}
            \item 第1章(本章)では,研究の背景と目的について述べた.
            \item 第2章「本研究におけるブラキエーション動作と実機構成」では,本研究で目標とするブラキエーション動作と,
            伸縮する単リンクブラキエーションロボットの実機構成について述べる.
            \item 第3章「最適なバーリリース条件の導出」ではバーの位置に基づく最適なバーリリース条件の導出とリリース実験の結果について述べる.
            \item 第4章「励振制御」ではXXX.
            \item 第5章「空中過程を含むブラキエーション実験」ではXXX.
            \item 第6章「結論および今後の展望」ではXXX.
          \end{itemize}


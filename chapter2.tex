\chapter[本研究におけるブラキエーション動作と実機構成]%
        {本研究におけるブラキエーション動作と実機構成}
        \section{はじめに}
        \seclabel{2-1}
          
          本章では,目的とするブラキエーション動作と伸縮による励振,そして本研究で用いるロボットの
          実機構成について述べる.


        \section{ブラキエーションの流れ}

          \figref{brachiationFig.png}に本研究で目的とする空中過程を含んだブラキエーション動作を示す.
          ロボットの両端のグリッパーがバーを掴んだ状態(\figref{brachiationFig.png}(1))から,
          片方のバーを離して振子過程(\figref{brachiationFig.png}(2))に移る.
          その後,適切なタイミングでもう一方のバーも離し(\figref{brachiationFig.png}(3)),空中過程を経て,
          最後に目標のバーを掴む(\figref{brachiationFig.png}(4)).これらの動きを繰り返すことで連続したブラキエーションを行う.

        \section{伸縮による励振}
          
          目標とするバーの位置が,把持していたバー(\figref{brachiationFig.png}におけるbar1)と同じ,もしくはそれよりも高い場合,
          振子過程においてロボットの振幅を大きくしなければ空気抵抗や摩擦などの影響により目標のバーへ到達することができない.
          そのため,振子過程において外部からのエネルギー投入による振動の拡大が望まれる.
          本研究ではロボットを伸縮することにより,ブランコのように重心位置をロボットの長手方向に変化させることで励振を行う.
          この励振方法は先行研究があり,実機で実現されている\cite{Hijiri:Robomech2024}.
          
          \figb{brachiationFig.png}{width=1\hsize}{Brachiation motions}
        
        \section{伸縮する単リンクブラキエーションロボットのシステムと実機構成}
          
          本研究で用いた実機は\cite{Hijiri:Robomech2024}


        
        
        \section{}
        \section{自在移動のためのリリース条件と励振制御}
        
          実験的にバーを離すタイミングを決めることもできるが、
          バーの座標が変わった時に
          実用的ではない

\chapter[バーの位置に基づく空中過程を含まないブラキエーション動作]%
{バーの位置に基づく\\空中過程を含まないブラキエーション動作}
        \section{はじめに}
        
        伸縮する単リンクブラキエーションロボットは,その「伸縮する」という特徴を活用することで,
        目標のバーまでの距離に合わせてリンクの長さを調整して移動することが可能となる.
        これにより,おもりを動かすことで励振するブラキエーションロボット\cite{akahane2022single}よりも
        自在な移動が実現できる.
        先行研究\cite{Hijiri:Robomech2024}では実験的にリンクを伸ばすタイミングを決定していた.
        しかし,伸ばすタイミングや長さが適切ではない場合,バーを把持できないだけでなく,
        バーとロボットが衝突することで不具合が生じたり,破損してしまう.
        また,目標バーを異なる位置にした場合,再び実験的にリンクを伸ばすタイミングを模索しなければならない.
        そこで,本章ではバーの位置に基づいた空中過程を含まないブラキエーション動作について述べる.
          
        \section{伸縮タイミングの検討}

        リンクを伸ばして次のバーを把持する場合,伸ばすタイミングは\figref{noAerialBrachiationFig.eps}に示すように以下の4通りの方法が考えられる.

        \begin{itemize}
        \item 方法1 (\figref{noAerialBrachiationFig.eps}(1)):
        
        バーに対して下側から近づき,バーに近づいたらリンクを伸ばし始める.

        バー把持時のグリッパーに対する回転方向の負荷が小さいが,
        伸縮による角速度の変化や,伸縮時間を考慮する必要があり伸縮制御が複雑になる.
        
        \item 方法2 (\figref{noAerialBrachiationFig.eps}(2)):
        
        バーに対して上側から近づき,バーに近づいたらリンクを伸ばし始める.
        
        バー把持時のグリッパーに対する回転方向の負荷が小さいが,
        伸縮による角速度の変化や,伸縮時間を考慮する必要があり伸縮制御が複雑になる.

        \item 方法3 (\figref{noAerialBrachiationFig.eps}(3)):
        
        最高点においてリンクを目標の長さまで伸ばし終えた状態で上側からバーに近づく.
        
        振子の最高点に達したときに伸縮するため,伸縮制御が容易であるが,
        最高点が目標のバーから離れている場合,バー把持時のグリッパーに対する回転方向の負荷が大きくなる.
        
        \item 方法4 (\figref{noAerialBrachiationFig.eps}(4)):
        
        最高点においてリンクを目標の長さまで伸ばし終えた状態で下側からバーに近づく.

        方法3と同様に伸縮制御が容易であるが,
        把持していたバー(\figref{noAerialBrachiationFig.eps}におけるbar 1)と衝突する可能性がある.

        \end{itemize}

        上記のうち,方法3は振子の最高点が目標のバーから離れすぎない位置まで励振したタイミングでリンクを伸ばすことで,
        バー把持時のグリッパーに対する回転方向の負荷を小さくすることが可能であるため,本研究では方法3を採用した.


        \fig{noAerialBrachiationFig.eps}{width=1.0\hsize}{Approach methods in brachiation movements without an aerial phase}

        \section{目標バーの位置に基づくリンクの長さ調整}

        \figref{BarPositionFig.eps}に目標とするバーとロボットの位置の概略図を示す.
        座標軸は左向きを$x$軸の正方向,上向きを$z$軸の正方向に設定し,
        ロボットが把持しているバーの座標を原点$(0,0)$,
        目標バーの座標を$(l_{\mathrm{bx}},l_{\mathrm{bz}})$とし,原点と結んだ線分が$z$軸となす角度を$\theta$とする.
        また,ロボットは角度$\varphi$,角速度$\dot{\varphi}$とボディの全長$l_{\mathrm{r}}$で表す.
        グリッパーが届く範囲にあるバーであれば,空中過程を含まないブラキエーションが可能である.
        ゆえに,ロボットの最大の長さを$l_{\mathrm{rMax}}$,最小の長さを$l_{\mathrm{rMin}}$とすると
        目標とするバーの条件は\equref{barRange}となる.
        また,目標とするバーの座標を用いて角度$\theta$は\equref{theta}で表される.
        \begin{eqnarray}
                \equlabel{barRange}
                l_{\mathrm{rMin}} &\le& \sqrt{l_{\mathrm{bx}}^2+l_{\mathrm{bz}}^2} \le l_{\mathrm{rMax}}\\
                \equlabel{theta}
                \theta&=&90+\arctan{\left(\frac{l_{\mathrm{bz}}}{l_{\mathrm{bx}}}\right)}
                \end{eqnarray}  
        

        thetaが
        励振の最大長はバーまでの距離?
        180に近づくとtheta+30を最大振幅φが超えたらというのはできなくなる
        範囲狭めるか、励振でジャストは難しいよね、この後の励振制御次第

        \begin{eqnarray}
                \equlabel{lr-range}
                x_{\mathrm{c}}&=&\frac{1}{2}l_{\mathrm{r}}\dot{\varphi}\cos{(\varphi)}t+\frac{1}{2}l_{\mathrm{r}}\sin{(\varphi)}\\
                \equlabel{bar}
                z_{\mathrm{c}}&=&\frac{1}{2}l_{\mathrm{r}}\dot{\varphi}\sin{(\varphi)}t-\frac{1}{2}gt^2-\frac{1}{2}l_{\mathrm{r}}\cos{(\varphi)}\\
                \equlabel{x-e}
                x_{\mathrm{e}}&=&x_{\mathrm{c}}+\frac{1}{2}l_{\mathrm{r}}\sin{(\varphi+\dot{\varphi}t)}\\
                \equlabel{z-e}
                z_{\mathrm{e}}&=&z_{\mathrm{c}}-\frac{1}{2}l_{\mathrm{r}}\cos{(\varphi+\dot{\varphi}t)}\\
                \equlabel{Jd}
                  J_{\mathrm{d}}(\varphi,\dot{\varphi},t,l_{\mathrm{r}})
                  &=&\sqrt{(l_{\mathrm{bx}}-x_{\mathrm{e}})^2+(l_{\mathrm{bz}}-z_{\mathrm{e}})^2}\\
                \equlabel{Jr}
                J_{\mathrm{r}}(\varphi,\dot{\varphi},t,l_{\mathrm{r}})
                &=&\sqrt{\dot{x_{\mathrm{e}}}^2+\dot{z_{\mathrm{e}}}^2}
              \end{eqnarray}  


        \fig{BarPositionFig.eps}{width=0.8\hsize}{Schematic Diagram}


        ここにバー座標の図、座標から距離、角度、移動可能な範囲を半円で示したり(角速度はいらないかな)

        その次に上から掴むか、下からか、どこで伸ばすかの議論。

        \section{空中過程を含まないブラキエーション動作の実機実験}



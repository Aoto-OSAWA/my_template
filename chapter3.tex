\chapter[最適なバーリリース条件の導出]%
{最適なバーリリース条件の導出}
\chaplabel{chapter3}
        \section{はじめに}
        \seclabel{3-1}

        空中過程を含むブラキエーション動作は,目標とするバーを把持することができなかった場合に落下してしまうという危険性がある.
        確実なバー把持のための条件には,バーとグリッパーの距離に加え,バーとの衝突を考慮することも望まれる.
        そこで,本研究では目標バーとロボットのグリッパー間の距離と,バー把持時のバーに対するグリッパーの相対速度を用いた評価関数によって
        バーリリース条件を最適化することを提案する.
        本章では,任意のバーの位置に基づくリリース条件最適化と,最適条件を基に行ったリリース実験について述べる.

        \section{空中過程における目標バーとグリッパーの距離と相対速度の導出}
        
          \figref{}にバーリリース時の模式図を示す.
          座標軸は左向きを$x$軸の正方向,上向きを$z$軸の正方向に設定し,
          ロボットが把持しているバーの座標を原点$(0,0)$,
          目標バーの座標を$(l_{\mathrm{bx}},l_{\mathrm{bz}})$とする.
          また,ロボットは姿勢$\varphi$とボディの全長$l_{\mathrm{r}}$の2変数を持つ.
          ここで,バーをリリースした後の空中過程におけるロボットの長さ$l_{\mathrm{r}}$は
          バーリリース時から変えずに一定であるとすると, 
          ロボットの重心の軌道はバーリリース時の角度$\varphi$,角速度$\dot{\varphi}$による斜方投射の放物線軌道,
          手先の軌道はバーリリース時の角速度$\dot{\varphi}$による重心周りの一定速回転軌道となる.
          ゆえに,バーリリースから$t$秒後の目標バーとグリッパーとの距離$J_{\mathrm{d}}$,目標バーに対するグリッパーの相対速度$J_{\mathrm{r}}$は
          それぞれ\equref{Jd},\equref{Jr}で表される.
          ここで,$g$は重力加速度,$(x_{\mathrm{c}},z_{\mathrm{c}})$,$(\dot{x_{\mathrm{c}}},\dot{z_{\mathrm{c}}})$はロボットの重心の位置と速度,
          $(x_{\mathrm{e}},z_{\mathrm{e}})$,$(\dot{x_{\mathrm{e}}},\dot{z_{\mathrm{e}}})$はグリッパーの手先の位置と速度を表す.
          \begin{eqnarray}
            \equlabel{x-c}
            x_{\mathrm{c}}&=&\frac{1}{2}l_{\mathrm{r}}\dot{\varphi}\cos{(\varphi)}t+\frac{1}{2}l_{\mathrm{r}}\sin{(\varphi)}\\
            \equlabel{z-c}
            z_{\mathrm{c}}&=&\frac{1}{2}l_{\mathrm{r}}\dot{\varphi}\sin{(\varphi)}t-\frac{1}{2}gt^2-\frac{1}{2}l_{\mathrm{r}}\cos{(\varphi)}\\
            \equlabel{x-e}
            x_{\mathrm{e}}&=&x_{\mathrm{c}}+\frac{1}{2}l_{\mathrm{r}}\sin{(\varphi+\dot{\varphi}t)}\\
            \equlabel{z-e}
            z_{\mathrm{e}}&=&z_{\mathrm{c}}-\frac{1}{2}l_{\mathrm{r}}\cos{(\varphi+\dot{\varphi}t)}\\
            \equlabel{Jd}
              J_{\mathrm{d}}(\varphi,\dot{\varphi},t,l_{\mathrm{r}})
              &=&\sqrt{(l_{\mathrm{bx}}-x_{\mathrm{e}})^2+(l_{\mathrm{bz}}-z_{\mathrm{e}})^2}\\
            \equlabel{Jr}
            J_{\mathrm{r}}(\varphi,\dot{\varphi},t,l_{\mathrm{r}})
            &=&\sqrt{(\dot{x_{\mathrm{e}}}-0)^2+(\dot{z_{\mathrm{e}}}-0)^2}\\
          \end{eqnarray}  
          % \begin{eqnarray}
          %   \equlabel{x-c-dot}
          %   \dot{x_{\mathrm{c}}}=
          %   \end{eqnarray}  
          % \begin{eqnarray}
          %   \equlabel{z-c-dot}
          %   \dot{z_{\mathrm{c}}}=
          %   \end{eqnarray}
    
          % \begin{eqnarray}
          %   \equlabel{x-z}
          %   \begin{split}
          %     x_{\mathrm{e}}=x_{\mathrm{c}}(t)+l_{\mathrm{r}}\sin(\varphi+\dot{\varphi}t)/2\\
          %     z_{\mathrm{e}}=z_{\mathrm{c}}(t)-l_{\mathrm{r}}\cos(\varphi+\dot{\varphi}t)/2
          %     \end{split}
          %   \end{eqnarray}


          
        

          \section{最適なバーリリース条件に基づくリリース実験}


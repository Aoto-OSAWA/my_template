\chapter[本研究におけるブラキエーション動作と実機構成]%
        {本研究におけるブラキエーション動作と\\実機構成}
        \section{はじめに}
          
          本章では,目的とするブラキエーション動作と伸縮による励振,そして本研究で用いるロボットの
          実機構成について述べる.


        \section{ブラキエーションの流れ}

          \figref{brachiationFig-4.eps}に本研究で目的とする空中過程を含んだブラキエーション動作を示す.
          ロボットの両端のグリッパーがバーを掴んだ状態(\figref{brachiationFig-4.eps}(1))から,
          片方のバーを離して振子過程(\figref{brachiationFig-4.eps}(2))に移る.
          目標のバーまでの距離がロボットの最大長以下である場合,バーの距離に合わせてロボットを伸縮させることで
          空中過程を含まないブラキエーション(\figref{brachiationFig-4.eps}(3-1))を行う.
          一方,目標のバーまでの距離がロボットの最大長以上である場合,適切なタイミングでもう一方のバーも離し(\figref{brachiationFig-4.eps}(3-2)),
          空中過程を経て,最後に目標のバーを掴む(\figref{brachiationFig-4.eps}(4)).
          これらの動きを繰り返すことで連続したブラキエーションを行う.

        \section{伸縮による励振}
          
          目標とするバーの位置が,把持していたバー(\figref{brachiationFig-4.eps}におけるbar1)と同じ,もしくはそれよりも高い場合,
          振子過程においてロボットの振幅を大きくしなければ空気抵抗や摩擦などの影響により目標のバーへ到達することができない.
          そのため,振子過程において外部からのエネルギー投入による振動の拡大が望まれる.
          本研究ではロボットを伸縮することにより,ブランコのように重心位置をロボットの長手方向に変化させることで励振を行う.
          この励振方法は先行研究があり,実機で実現されている\cite{Hijiri:Robomech2024}.
          
          \figb{brachiationFig-4.eps}{width=0.9\hsize}{Brachiation motions}
        
          % \newpage
        \section{伸縮する単リンクブラキエーションロボットの実機構成}
          
          本研究では\cite{Hijiri:Robomech2024}で使用していた実機を用いた.
          \figref{bar-robot4.png}に全体図を示す.重さは3.0 kgであり,幅・奥行・高さは
          最も縮めた場合は$150{\times}80{\times}560$ mm,最も伸ばした場合は$150{\times}80{\times}740$ mmである.
          伸縮にはラックピニオン機構が用いられており,中心部のブラシレスDCモータ(MAXON EC22 40W, ギアヘッドギア比128)によって歯車を回転させる.
          ブラシレスDCモータはモータドライバ(EPOS2 24/5)を介してロボット全体の制御を行うArduino Unoに接続されている.
          また,グリッパーは把持部品がサーボモジュール(双葉電子工業 RS406CB)で駆動する.
          なお,グリッパーに用いているサーボモータの回転速度は0.11s/60degであり,
          グリッパー開状態は0 deg,グリッパー閉状態は100 degであるため
          グリッパーを開く/閉じるために必要なモータ回転量100 degの回転に要する時間は
          $100/60\times0.11\approx$0.183 sである.
          ロボット全体の姿勢取得にはIMU(Adafruit BNO055)を用いている.
          
          \fig{photo4.jpg}{width=0.5\hsize}{Overall view of extensible brachiation robot}

            
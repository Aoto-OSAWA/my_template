\chapter[最適なバーリリース条件の導出]%
{最適なバーリリース条件の導出}
\chaplabel{chapter3}
      \section{はじめに}
      \seclabel{3-1}

      空中過程を含むブラキエーション動作は,目標とするバーを把持することができなかった場合に落下してしまうという危険性がある.
      確実なバー把持のための条件には,バーとグリッパーの距離に加え,バーとの衝突を考慮することも望まれる.
      そこで,本研究では目標バーとロボットのグリッパー間の距離と,バー把持時のバーに対するグリッパーの相対速度に基づく評価関数を用いて
      バーリリース条件を最適化することを提案する.
      本章では,任意のバーの位置に基づくリリース条件最適化と,最適条件を基に行ったリリース実験について述べる.

      \section{最適なバーリリース条件の導出}
        \subsection{空中過程における目標バーとグリッパーの距離と相対速度}

          \figref{ReleaseFig2.eps}に示したロボットのバーリリース時の状態から,
          空中過程における目標バーとグリッパーの距離,相対速度を導出する.
          座標軸は左向きを$x$軸の正方向,上向きを$z$軸の正方向に設定し,
          ロボットが把持しているバーの座標を原点$(0,0)$,
          目標バーの座標を$(l_{\mathrm{bx}},l_{\mathrm{bz}})$とする.
          また,ロボットは姿勢$\varphi$とボディの全長$l_{\mathrm{r}}$の2変数を持つ.         
          ここで,バーリリース後の空中過程におけるロボットの長さ$l_{\mathrm{r}}$は
          バーリリース時から変えずに一定であるとすると, 
          ロボットの重心の軌道はバーリリース時の角度$\varphi$,角速度$\dot{\varphi}$による放物線軌道,
          手先の軌道はバーリリース時の角速度$\dot{\varphi}$による重心周りの一定速回転軌道となる.
          ゆえに,バーリリースから$t$秒後の目標バーとグリッパーとの距離$J_{\mathrm{d}}$,目標バーに対するグリッパーの相対速度$J_{\mathrm{r}}$は
          それぞれ\equref{Jd},\equref{Jr}で表される.
          ここで,$g$は重力加速度,$(x_{\mathrm{c}},z_{\mathrm{c}})$,$(\dot{x_{\mathrm{c}}},\dot{z_{\mathrm{c}}})$はロボットの重心の位置と速度,
          $(x_{\mathrm{e}},z_{\mathrm{e}})$,$(\dot{x_{\mathrm{e}}},\dot{z_{\mathrm{e}}})$はグリッパーの手先の位置と速度を表す.
          \begin{eqnarray}
            \equlabel{x-c}
            x_{\mathrm{c}}&=&\frac{1}{2}l_{\mathrm{r}}\dot{\varphi}\cos{(\varphi)}t+\frac{1}{2}l_{\mathrm{r}}\sin{(\varphi)}\\
            \equlabel{z-c}
            z_{\mathrm{c}}&=&\frac{1}{2}l_{\mathrm{r}}\dot{\varphi}\sin{(\varphi)}t-\frac{1}{2}gt^2-\frac{1}{2}l_{\mathrm{r}}\cos{(\varphi)}\\
            \equlabel{x-e}
            x_{\mathrm{e}}&=&x_{\mathrm{c}}+\frac{1}{2}l_{\mathrm{r}}\sin{(\varphi+\dot{\varphi}t)}\\
            \equlabel{z-e}
            z_{\mathrm{e}}&=&z_{\mathrm{c}}-\frac{1}{2}l_{\mathrm{r}}\cos{(\varphi+\dot{\varphi}t)}\\
            \equlabel{Jd}
              J_{\mathrm{d}}(\varphi,\dot{\varphi},t,l_{\mathrm{r}})
              &=&\sqrt{(l_{\mathrm{bx}}-x_{\mathrm{e}})^2+(l_{\mathrm{bz}}-z_{\mathrm{e}})^2}\\
            \equlabel{Jr}
            J_{\mathrm{r}}(\varphi,\dot{\varphi},t,l_{\mathrm{r}})
            &=&\sqrt{\dot{x_{\mathrm{e}}}^2+\dot{z_{\mathrm{e}}}^2}
          \end{eqnarray}  
        
          
        % \fig{barReleasefig.png}{width=0.6\hsize}{Schematic Diagram}
        \fig{ReleaseFig2.eps}{width=0.7\hsize}{Schematic Diagram}
        
        \subsection{最適化のための評価関数}
        \seclabel{J-define}
        
        バーを離すときの角度$\varphi$, 角速度$\dot{\varphi}$,ロボットの長さ$l_{\mathrm{r}}$の3つの値をバーリリース条件とする.
        バーリリース条件の最適化のために,目標バーとグリッパーの距離と相対速度がともに小さくなる条件を求める.
        ここで,評価関数$J$を\equref{Jd},\equref{Jr}で示した距離$J_{\mathrm{d}}$と相対速度$J_{\mathrm{r}}$
        を用いて\equref{J}とした.
        ここで,$\alpha$は重み係数を表し,確実な把持のためにバーとグリッパーの距離に重みづけを行う.
        \begin{eqnarray}
          \equlabel{J}
          J(\varphi,\dot{\varphi},t,l_{\mathrm{r}})=\alpha{\times}J_{\mathrm{d}}+J_{\mathrm{r}}
        \end{eqnarray}
        評価関数が最小になるとき,目標とするバーとグリッパーの距離と相対速度がともに小さくなる.
        ゆえに,評価関数を最小にする角度$\varphi$, 角速度$\dot{\varphi}$,ロボットの長さ$l_{\mathrm{r}}$,
        リリース後のバー把持までの時間$t$を導出し,そのうち角度,角速度,ロボットの長さを最適なバーリリース条件とした.
        また,時刻はグリッパーを閉じるタイミングの決定に用いた.
        評価関数の最小化にはニュートン法と共役勾配法を組み合わせたNewton-CG法を用いた.

      \section{最適なバーリリース条件に基づくリリース実験}
      
      \secref{J-define}に基づいて最適なバーリリース条件を求め,
      実機を用いてリリース実験を行った.  
      
        \subsection{実験条件と最適なバーリリース条件}
        \begin{table}[tb]
          
          \begin{center}
            \caption{Experiment conditions}
            \tablabel{ExperimentConditions}
            \vspace{2mm}
            \begin{tabular}{c|c}
              \hline
              Variables & Values \\
              \hline
              $l_{\mathrm{bx}}$ [m] & 0.79 \\
              $l_{\mathrm{bz}}$ [m] & 0.00 \\
              $\alpha$ [-]& 20 \\
              $g$ $\mathrm{[m/s^2]}$ & 9.81 \\
              \hline
            \end{tabular}
          \end{center}
        \end{table}
  
        \begin{table}[tb]
          
          \begin{center}
            \caption{Optimized conditions values}
            \tablabel{optimizedRelease}
            \vspace{2mm}
            \begin{tabular}{c|c}
              \hline
              Variables & Values \\
              \hline
              $\varphi$ [deg] & 56 \\
              $\dot{\varphi}$ [deg/s] & 260 \\
              $l_{\mathrm{r}}$ [m] & 0.680 \\
              $t$ [s] & 0.261 \\
              $J$ [-] & 0.0144 \\
              $J_{\mathrm{d}}$ [m] & 0.00115 \\
              $J_{\mathrm{r}}$ [m/s] & 0.00292 \\
              \hline
            \end{tabular}
          \end{center}
        \end{table}
        実験条件を\tabref{ExperimentConditions}に示す.
        また,その実験条件を基に求めた最適なバーリリース条件と,評価値を\tabref{optimizedRelease}に示す.
        実験では,ロボットの長さは導出した条件で固定されるようにブラシレスモータを制御し,
        導出した角度・角速度になる初期振幅を実験的に求めた.
        バーリリースの指令は,IMUで角度をリアルタイムに計測し,リリース条件の角度を満たした時にグリッパーが完全に開いた状態になるように
        グリッパーのサーボモータの回転速度を考慮して設定した.
        グリッパーを閉じるタイミングは,導出した時刻を基に,サーボモータの回転速度を考慮して決定し,
        空中過程における空気抵抗なども考慮し,実験的に調整を行った.

        \subsection{実機実験}
        実験の様子を\figref{}に示す.
        


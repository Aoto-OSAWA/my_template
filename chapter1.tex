\chapter[序論]%
        {序論}
        \section{研究の背景と目的}
        \seclabel{1-1}

          ブラキエーションは,上肢で枝を掴んでぶら下がりながら移動する方法であり,重力を利用することで高所を効率的に移動できる.
          この移動方法をロボットに応用することで\cite{福田敏男1990ブラキエーション形移動ロボットの研究},送電線の点検などの高所作業への適用が期待される.
          テナガザルを模倣した多リンク型のロボットの研究例として,福田らの2リンク型\cite{福田敏男1991ブラキエーション形移動ロボットの研究2}\cite{福田敏男1992ブラキエーション形移動ロボットの研究}\cite{齋藤史倫1993ブラキエーション形移動ロボットの研究}\cite{齋藤史倫1995学習とロボット}\cite{福田敏男1996強化学習法を用いたファジィコントローラの生成}\cite{中西淳1998解析的手法による}\cite{中西淳19992}\cite{中西淳2001ハイブリッドコントローラによる}や
          5リンク型\cite{福田敏男1991ブラキエーション形移動ロボットの研究},6リンク型\cite{福田敏男1990ブラキエーション形移動ロボットの研究},7リンク型\cite{齋藤史倫1994ブラキエーション形移動ロボットの研究},13リンク型\cite{長谷川泰久2001ブラキエーション形移動ロボットの研究}などがある.
          また,把持機構に電磁石を用いた2リンク型\cite{山川雄司2016ブラキエーションロボットの開発と運動生成}\cite{山川雄司2016-2ブラキエーションロボットの開発と運動生成}や,
          パッシブグリッパーを用いた2リンク型\cite{javadi2023acromonk},3リンク型\cite{grama2024ricmonk}などがある.
          しかし,多リンク型は構造が複雑であるとともに,
          カオス現象\cite{鈴木三男2000二重振り子におけるカオス的振舞}が生じることで制御が難しくなるという問題がある.
          赤羽らはロボットの形状を棒状,すなわち単リンク型にすることで構造を単純化し,これらの問題を解決した\cite{akahane2022single}.
          また,おもりを動かすことで重心を移動させて励振する手法\cite{lieskovsky2023optimal}を用いたロボットの,モデル予測制御\cite{Hijiri:Robomech2024-1}によるブラキエーションの研究や,
          伸縮することで励振するとともに移動可能距離を延ばす\cite{Hijiri:Robomech2024}などの単リンクブラキエーションロボットに関する研究がある.
          これらの単リンクブラキエーションロボットの研究では,励振量を調整を行っておらず,さらに伸縮するブラキエーションロボットにおいては実験的に伸縮タイミングを決定しており,
          目標のバーの位置に基づいた最適な移動までには至っていない.また,同じ高さのバーへの移動のみであり,異なる高さのバーへの移動は実現されていない.
          
          そこで本研究では,伸縮する単リンクブラキエーションロボットの自在移動を実現することを目的とし,バーの位置が異なる場合でも適用できるシステムを提案・検証する.
          空中過程を含むブラキエーション動作では,目標のバーの位置に基づいた最適なバーリリース条件を導出し,伸縮する機構を活かした励振調整により条件を実現させる.
          また,空中過程を含まないブラキエーション動作においても,先行研究では励振調整を行わず,さらに実験的に伸縮タイミングを決定していたが\cite{Hijiri:Robomech2024},
          本研究ではほほ目標のバーの位置に基づいて励振調整・伸縮タイミング決定を行う.
          
        \section{本論文の構成}
        \seclabel{structure}

          本論文は,全7章から構成させる.以下に各章の概要を述べる.
          \begin{itemize}
            \item 第1章(本章)では,研究の背景と目的について述べた.
            \item 第2章「本研究におけるブラキエーション動作と実機構成」では,本研究で目標とするブラキエーション動作と,
            伸縮する単リンクブラキエーションロボットの実機構成について述べる.
            \item 第3章「最適なバーリリース条件の導出」では,空中過程を含むブラキエーションにおけるバーの位置に基づく最適なバーリリース条件の導出とリリース実験について述べる.
            \item 第4章「リリース条件到達のための励振調整」では,最適なバーリリース条件を実現させるための,振子過程における伸縮量制御による励振調整について述べる.
            \item 第5章「励振調整を用いたブラキエーション実験」では最適なバーリリースによる空中過程を含むブラキエーション動作の実験について述べる.
            \item 第6章「バーの位置に基づく空中過程を含まないブラキエーション動作」では,バーの位置に基づいて空中過程を含まないブラキエーション動作を行うための
            伸縮量制御システムと実機実験について述べる.
            \item 第7章「結論および今後の展望」では,本研究で得られた結論および今後の展望について述べる.
          \end{itemize}

          
          %   \item 第3章「最適なバーリリース条件の導出」では,空中過程を含むブラキエーションにおけるバーの位置に基づく最適なバーリリース条件の導出とリリース実験について述べる.
          %   \item 第4章「励振調整」では,最適なバーリリース条件を実現させるための振子過程における励振調整について述べる.
          %   \item 第5章「空中過程を含むブラキエーション実験」では最適なバーリリースによる空中過程を含むブラキエーション動作の実験について述べる.
          %   \item 第6章「バーの位置に基づく空中過程を含まないブラキエーション動作」では,バーの位置に基づいて空中過程を含まないブラキエーション動作を行うための
          %   伸縮量制御システムの提案と実機実験について述べる.
          %   \item 第7章「結論および今後の展望」では,本研究で得られた結論および今後の展望について述べる.


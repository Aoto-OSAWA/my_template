\chapter[結論および今後の展望]%
{結論および今後の展望}
        \section{結論}

        本研究では,伸縮する単リンクブラキエーションロボットが2次元空間を自在に移動することを目標とした.
        まず,空中過程を含むブラキエーション動作のために,
        目標のバーの位置に基づく最適なバーリリース条件を把持時の距離と相対速度を基に導出した.
        また,その条件を実現させるために必要な振幅を求め,伸縮する機構を活かした励振調整システムを提案し,
        実機実験により励振調整を用いた最適なバーリリースによる,空中過程を含む目標バーへの到達と目標バーの把持に成功した.
        さらに,空中過程を含まないブラキエーション動作においても,目標バーの位置に基づいて目標振幅を決定し,
        提案した励振調整システムを基に目標振幅を実現し,目標バーの把持に成功した.

          

        \section{今後の展望}
        
        今後の展望として,移動可能範囲の拡張やロバスト性向上などがある.
        本研究では目標とするバーを掴むグリッパーを片方のみに限定して研究を行ったが,移動前に最初のバーを把持しているグリッパーの空中過程での軌道を考慮することで,
        把持可能な範囲が広がる可能性がある.
        また,3次元方向に移動することができる機構にすることにより,本研究で行った2次元方向の移動よりも自在な移動が可能になる.
        さらに,目標とするバーを認識する機構などを用いることで,より正確な把持によるロバスト性向上やバーの位置を認識して自律的に移動することも期待できる.
        また,本研究で提案した最適なバーリリース条件でリリースした場合でも,理想的な軌道からずれてしまう可能性がある.
        そのため,空中過程でも伸縮制御を行うなどして軌道修正をすることでロバスト性を向上させられると考える.

\documentclass{article}

\begin{document}

\title{Paper Introduction}
\author{Aoto Osawa}
\date{2024-11-01}
\maketitle

\begin{abstract}
% abstractをここに書く(下はダミーテキスト)
This paper presents the design, analysis, and performance evaluation of RicMonk, a novel three-link brachiation robot equipped with passive hook-shaped grippers. Brachiation, an agile and energy-efficient mode of locomotion observed in primates, has inspired the development of RicMonk to explore versatile locomotion and maneuvers on ladder-like structures. The robot’s anatomical resemblance to gibbons and the integration of a tail mechanism for energy injection contribute to its unique capabilities. The paper discusses the use of the Direct Collocation methodology for optimizing trajectories for the robot’s dynamic behaviors and stabilization of these trajectories using a Time-varying Linear Quadratic Regulator. With RicMonk we demonstrate bidirectional brachiation, and provide comparative analysis with its predecessor, AcroMonk - a two-link brachiation robot, to demonstrate that the presence of a passive tail helps improve energy efficiency. The system design, controllers, and software implementation are publicly available on GitHub1. Index Terms—Underactuated robots, biologically-inspired robots, education robotics.\cite{grama2024ricmonk}
\end{abstract}

% 参考文献をbibtexで処理
\bibliographystyle{unsrt}
\bibliography{reference}

\end{document}

\chapter[リリース条件到達のための励振制御]%
{リリース条件到達のための励振制御}
        \section{はじめに}

        \chapref{chapter4}において導出した最適なバーリリース条件を実現するためには,
        振子過程における励振調整が望まれる.
        本章では伸縮タイミングと伸縮量の調整による目標角度・角速度の実現について述べる.
          
        \section{伸縮による励振のメカニズム}

          本研究では,重心を移動させることにより振子過程において励振させる.
          Lieskovsk{\`y}らの重りを動かすことによる振幅増幅率が最大となる重心移動の研究\cite{lieskovsky2023optimal}を,
          伸縮機構に応用して励振を行う.
          
          \subsection{伸縮する単リンクブラキエーションロボットのモデル化}
          \figref{modelFig.eps}に伸縮する単リンクブラキエーションロボットを簡略化した,
          伸縮する振子のモデルを示す.振子の半径方向と鉛直下向き線がなす角を$\varphi [\mathrm{rad}]$,
          振子の長さを$l [\mathrm{m}]$,振子の質量を$m [\mathrm{kg}]$,振子の重心周りの慣性モーメントを$I [\mathrm{kg}\mathrm{m}^2]$,
          重力加速度を$g [\mathrm{m}/\mathrm{s}^2]$とする.
          \fig{modelFig.eps}{width=0.3\hsize}{Model of an extensible pendulum}
          運動方程式を以下のラグランジュの運動方程式で求める.
          \begin{eqnarray}
            \equlabel{Lagrange}
            \frac{\mathrm{d}}{\mathrm{d}t}\left(\frac{\partial L}{\partial \dot{q_i}}\right)-\frac{\partial L}{\partial q_i}=S_i
            \end{eqnarray} 
          $q_i$と$S_i$はそれぞれ一般化座標と一般化力であり,それぞれ\equref{qq},\equref{SS}とした.
          ここで,伸縮のために加える力を$u[\mathrm{N}]$とする.
          $L=T-U$はラグランジアンであり,
          系の運動エネルギー$T$と位置エネルギー$U$で構成され,本モデルでは\equref{Tequ},\equref{U_equ}となる.
          \begin{eqnarray}
              \equlabel{qq}
              \begin{bmatrix}
                q_1 \\
                q_2 \\
                \end{bmatrix}
              &=&
              \begin{bmatrix}
                \varphi \\
                l \\
                \end{bmatrix}\\
              \equlabel{SS}
              \begin{bmatrix}
                S_1 \\
                S_2 \\
                \end{bmatrix}
              &=&
              \begin{bmatrix}
                0 \\
                u \\
                \end{bmatrix}
                \end{eqnarray}
              \begin{eqnarray}
              \equlabel{Tequ}
              T &=& \frac{1}{2} m \left(\frac{l}{2} \dot{\varphi} \right)^2 + \frac{1}{2} I \dot{\varphi}^2 + \frac{1}{2} m \dot{l}^2  \\
              \equlabel{U_equ}
              U &=& -gm\frac{l}{2}\cos(\varphi)
              \end{eqnarray}  
          運動方程式を行列で書き表すと\equref{Matrix}となる.
          \begin{eqnarray}
            \equlabel{Matrix}
            \underbrace{\frac{\partial^2 L}{\partial \dot{\bm{q}}^2}}_{\bm{M}} \ddot{\bm{q}}^2 +\underbrace{\frac{\partial^2 L}{\partial \dot{\bm{q}} \partial \bm{q}} \dot{\bm{q}} - \frac{\partial L}{\partial \bm{q}}}_{\bm{c}} = 
            \begin{bmatrix}
              0 \\
              u \\
              \end{bmatrix}
            \end{eqnarray} 
          ここで,$\bm{M}(l)$,$\bm{c}(\varphi,l,\dot{\varphi},\dot{l})$は以下のようになる.
            \begin{eqnarray}
              \equlabel{M}
              \bm{M}(l)&=&
              \begin{bmatrix}
                M_{11}(l) & 0\\
                0 & M_{22} \\
                \end{bmatrix}\\
              \equlabel{C}
              \bm{c}(\varphi,l,\dot{\varphi},\dot{l})&=&
              \begin{bmatrix}
                c_{1}(\varphi,l,\dot{\varphi},\dot{l}) \\
                c_{2}(\varphi,l,\dot{\varphi}) \\
                \end{bmatrix}\\
              \equlabel{M11}
              M_{11}(l) &=& \frac{1}{4}m{l(t)}^2 + I\\
              \equlabel{M22}
              M_{22} &=& \frac{m}{4}\\
              \equlabel{C1}
              c_{1}(\varphi,l,\dot{\varphi},\dot{l}) &=& \frac{1}{2}ml(t)\dot{l}(t)\dot{\varphi}(t) + \frac{1}{2}gml(t)\sin{(\varphi(t))}\\
              \equlabel{C2}
              c_{2}(\varphi,l,\dot{\varphi}) &=& -\frac{1}{4}ml(t){\dot{\varphi}(t)}^2 - \frac{1}{2}gm\cos{(\varphi(t))}
            \end{eqnarray} 
          本モデルの力積は\equref{Si}で計算され,系の運動量$p_i = \frac{\partial L}{\partial \dot{q}_i}$との関係は\equref{pi}である.
          \begin{eqnarray}
            \equlabel{Si}
            \hat{S}_i = \lim_{t^{+}\to t} \int_{t}^{t^{+}} S_i (\tau) \mathrm{d}\tau\\
            \equlabel{pi}
            p_i(t^{+})-p_i(t)=\hat{S}_i
            \end{eqnarray}
          本モデルにおいて,$\varphi$方向の運動量は保存されるため,\equref{impulse}が満たされる.
          \begin{eqnarray}
            \equlabel{impulse}
            M_{11}(l^{+})\dot{\varphi}^{+}-M_{11}(l)\dot{\varphi}=0
            \end{eqnarray}
          また,状態変数を
          \begin{eqnarray}
            \bm{x}=
            \begin{pmatrix}
              x_1\\
              x_2\\
              x_3\\
              x_4\\
              \end{pmatrix}
              =
              \begin{pmatrix}
                \varphi\\
                l\\
                \dot{\varphi}\\
                \dot{l}\\
                \end{pmatrix}
            \end{eqnarray}
          とすると,状態モデルは
          \begin{eqnarray}
            \dot{\bm{x}}=
            \begin{pmatrix}
              \dot{x}_1\\
              \dot{x}_2\\
              \dot{x}_3\\
              \dot{x}_4\\
              \end{pmatrix}
              =
              \begin{pmatrix}
                x_3\\
                x_4\\
                -M_{11}^{-1}(x_2)c_1(x_{1-4})\\
                -M_{22}^{-1}c_2(x_{1-3})+M_{22}^{-1}u\\
                \end{pmatrix}
            \end{eqnarray}
          となる.
          伸縮が時刻$t$秒から$t^{+}$秒の間に瞬間的に行われると仮定すると,
          伸縮後の状態$\bm{x}^{+}$は伸縮前の状態$\bm{x}$と\equref{impulse}を用いて
          \equref{xx}で表される.ここで,伸びる場合は$x_2^{+}=l_{\mathrm{max}}$,
          縮む場合は$x_2^{+}=l_{\mathrm{min}}$となり,\equref{lchange}の条件に基づいて変化させる.
          \begin{eqnarray}
            \equlabel{xx}
            \bm{x}^{+}&=&
            \begin{pmatrix}
                x_1\\
                x_2^{+}\\
                M_{11}^{-1}(x_2^{+})M_{11}(x_2)x_3\\
                0\\
                \end{pmatrix}\\
            \equlabel{lchange}
                l&=&\left\{
                l_{\mathrm{min}}\quad \mathrm{if}\,\varphi \dot{\varphi}>0\\
                l_{\mathrm{max}}\quad \mathrm{if}\,\varphi \dot{\varphi}<0\\
                \right\}
            \end{eqnarray}
          この瞬間的な伸縮により系の運動エネルギー$T$と位置エネルギー$U$は次のように変化する.
          \begin{eqnarray}
            \equlabel{dT}
            \Delta T_t^{t^{+}}&=&\left(\frac{M_{11}(x_2)}{M_{11}(x_2^{+})}-1\right)\frac{1}{2}M_{11}(x_2)x_3^2\\
            \equlabel{dU}
            \Delta U_t^{t^{+}}&=&gm(x_2^{+}-x_2)\cos{x_1}            
            \end{eqnarray}
          \subsection{励振シミュレーション}
          
          
          
          
          Lieskovsk{\`y}らのおもりを動かすことによる励振の研究\cite{lieskovsky2023optimal}を
          伸縮機構に応用し,励振のシミュレーションを行った.
          本モデルの運動を式で表すと
          \begin{eqnarray}
            \equlabel{FG}
            \left\{
              \begin{aligned}
                \dot{\bm{x}}= F(x),\quad\bm{x}\in C\\
                \bm{x}^{+}= G(x),\quad\bm{x}\in D
              \end{aligned}
            \right.
          \end{eqnarray} 
          となる.フロー集合$C$における$\bm{x}$の連続ダイナミクスと,ジャンプ集合$D$における$\bm{x}^{+}$の離散ダイナミクスを
          組み合わせたハイブリッドダイナミクスで運動を表す.ここで,$\bm{x}^{+}$は瞬間的な伸縮後の状態である.
          また,連続ダイナミクスでは$x_2\,:\,l$は一定値に維持される.
          ここで







            \begin{eqnarray}
              \equlabel{lr-range}
              x_{\mathrm{c}}&=&\frac{1}{2}l_{\mathrm{r}}\dot{\varphi}\cos{(\varphi)}t+\frac{1}{2}l_{\mathrm{r}}\sin{(\varphi)}\\
              \equlabel{bar}
              z_{\mathrm{c}}&=&\frac{1}{2}l_{\mathrm{r}}\dot{\varphi}\sin{(\varphi)}t-\frac{1}{2}gt^2-\frac{1}{2}l_{\mathrm{r}}\cos{(\varphi)}\\
              \equlabel{x-e}
              x_{\mathrm{e}}&=&x_{\mathrm{c}}+\frac{1}{2}l_{\mathrm{r}}\sin{(\varphi+\dot{\varphi}t)}\\
              \equlabel{z-e}
              z_{\mathrm{e}}&=&z_{\mathrm{c}}-\frac{1}{2}l_{\mathrm{r}}\cos{(\varphi+\dot{\varphi}t)}\\
              \equlabel{Jd}
                J_{\mathrm{d}}(\varphi,\dot{\varphi},t,l_{\mathrm{r}})
                &=&\sqrt{(l_{\mathrm{bx}}-x_{\mathrm{e}})^2+(l_{\mathrm{bz}}-z_{\mathrm{e}})^2}\\
              \equlabel{Jr}
              J_{\mathrm{r}}(\varphi,\dot{\varphi},t,l_{\mathrm{r}})
              &=&\sqrt{\dot{x_{\mathrm{e}}}^2+\dot{z_{\mathrm{e}}}^2}
            \end{eqnarray} 
        \section{振子過程での運動方程式}
        振子過程のシミュレーション説明
        2つの過程で調整を行う
          減衰
            数値解  
            解析解
            
            減衰データとフィッティング
          励振

          調整プログラム



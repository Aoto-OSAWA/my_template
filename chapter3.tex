\chapter[空中過程を含まないブラキエーション動作の伸縮制御]%
{空中過程を含まないブラキエーション動作の\\伸縮制御}
        \section{はじめに}
        
        伸縮する単リンクブラキエーションロボットは,その「伸縮する」という特徴を活用することで,
        目標のバーまでの距離に合わせて伸縮させて移動することが可能となる.
        これにより,おもりを動かすことで励振するブラキエーションロボット\cite{akahane2022single}よりも
        自在な移動が実現できる.
        先行研究\cite{Hijiri:Robomech2024}では実験的に伸縮やグリッパーを閉じるタイミングを決定していた.
        しかし,伸縮のタイミングや伸縮量が適切ではない場合,バーを把持できないだけでなく,
        バーとロボットが衝突することで不具合が生じたり,破損してしまう.
        そこで,本章では空中過程を含まないブラキエーション動作における,バーの位置に基づいた伸縮制御について述べる.
          
        \section{}

        ここにバー座標の図、座標から距離、角度、移動可能な範囲を半円で示したり(角速度はいらないかな)

        その次に上から掴むか、下からか、どこで伸ばすかの議論。

        


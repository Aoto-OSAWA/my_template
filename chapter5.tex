\chapter[伸縮量操作による振幅調整]%
{伸縮量操作による振幅調整}
        \section{はじめに}

        目標バーの位置に基づいて伸縮タイミングやリリースの条件を決定する場合,
        振子過程においてその条件を実現することが望まれる.
        本章では,伸縮量を操作することによる振幅調整システムについて述べる..
          
        \section{伸縮による励振のメカニズム}

          本研究では,重心を移動させることにより振子過程において励振させる.
          Lieskovsk{\`y}らの重りを動かすことによる振幅増加率が最大となる重心移動の研究\cite{lieskovsky2023optimal}を,
          伸縮機構に応用して励振を行う.
          
          \subsection{伸縮する棒状ブラキエーションロボットのモデル化}
          \figref{modelFig.eps}に伸縮する棒状ブラキエーションロボットを簡略化した,
          伸縮する剛体棒振子のモデルを示す.なお,グリッパーは質量を持たず,メインリンクは一様な一本の剛体棒であり,剛体棒振子の重心は剛体棒の中心にあるとした.
          振子の半径方向と鉛直下向き線がなす角を$\varphi [\mathrm{rad}]$,
          振子の長さを$l [\mathrm{m}]$,振子の質量を$m [\mathrm{kg}]$,振子の重心周りの慣性モーメントを$J [\mathrm{kg}\mathrm{m}^2]$,
          重力加速度を$g [\mathrm{m}/\mathrm{s}^2]$とする.
          \fig{modelFig.eps}{width=0.3\hsize}{Model of an extensible pendulum}
          運動方程式を以下のラグランジュの運動方程式で求める.
          \begin{eqnarray}
            \equlabel{Lagrange}
            \frac{\mathrm{d}}{\mathrm{d}t}\left(\frac{\partial L}{\partial \dot{q_i}}\right)-\frac{\partial L}{\partial q_i}=S_i
            \end{eqnarray} 
          $q_i$と$S_i$はそれぞれ一般化座標と一般化力であり,それぞれ\equref{qq},\equref{SS}とした.
          ここで,伸縮のために加える力を$u[\mathrm{N}]$とする.
          \begin{eqnarray}
            \equlabel{qq}
            \bm{q}&=&
            \begin{bmatrix}
              q_1 \\
              q_2 \\
              \end{bmatrix}
            =
            \begin{bmatrix}
              \varphi \\
              l \\
              \end{bmatrix}\\
            \equlabel{SS}
            \bm{S}&=&
            \begin{bmatrix}
              S_1 \\
              S_2 \\
              \end{bmatrix}
            =
            \begin{bmatrix}
              0 \\
              u \\
              \end{bmatrix}
              \end{eqnarray}
          $L=T-U$はラグランジアンであり,
          系の運動エネルギー$T$と位置エネルギー$U$で構成され,本モデルでは\equref{Tequ},\equref{U_equ}となる.
          \begin{eqnarray}
              \equlabel{Tequ}
              T &=& \frac{1}{2} m \left(\frac{l}{2} \dot{\varphi} \right)^2 + \frac{1}{2} J \dot{\varphi}^2 + \frac{1}{2} m \dot{l}^2  \\
              \equlabel{U_equ}
              U &=& -gm\frac{l}{2}\cos(\varphi)
              \end{eqnarray}  
          運動方程式を行列で書き表すと\equref{Matrix}となる.
          本モデル伸縮力$u$によって伸縮量$l(t)$が変化する,ノンホロノミックな系である.
          \begin{eqnarray}
            \equlabel{Matrix}
            \underbrace{\frac{\partial^2 L}{\partial \dot{\bm{q}}^2}}_{\bm{M}} \ddot{\bm{q}}^2 +\underbrace{\frac{\partial^2 L}{\partial \dot{\bm{q}} \partial \bm{q}} \dot{\bm{q}} - \frac{\partial L}{\partial \bm{q}}}_{\bm{c}} = \bm{S}
            \end{eqnarray} 
          ここで,$\bm{M}(l)$,$\bm{c}(\varphi,l,\dot{\varphi},\dot{l})$は以下のようになる.
            \begin{eqnarray}
              \equlabel{M}
              \bm{M}(l)&=&
              \begin{bmatrix}
                M_{11}(l) & 0\\
                0 & M_{22} \\
                \end{bmatrix}\\
              \equlabel{C}
              \bm{c}(\varphi,l,\dot{\varphi},\dot{l})&=&
              \begin{bmatrix}
                c_{1}(\varphi,l,\dot{\varphi},\dot{l}) \\
                c_{2}(\varphi,l,\dot{\varphi}) \\
                \end{bmatrix}\\
              \equlabel{M11}
              M_{11}(l) &=& \frac{1}{4}m{l(t)}^2 + J\\
              \equlabel{M22}
              M_{22} &=& \frac{m}{4}\\
              \equlabel{C1}
              c_{1}(\varphi,l,\dot{\varphi},\dot{l}) &=& \frac{1}{2}ml(t)\dot{l}(t)\dot{\varphi}(t) + \frac{1}{2}gml(t)\sin{(\varphi(t))}\\
              \equlabel{C2}
              c_{2}(\varphi,l,\dot{\varphi}) &=& -\frac{1}{4}ml(t){\dot{\varphi}(t)}^2 - \frac{1}{2}gm\cos{(\varphi(t))}
            \end{eqnarray} 
          伸縮が時刻$t$秒から$t^{+}$秒の間に瞬間的に行われると仮定する.ここで,長さは$l$から$l^{+}$,角度は$\phi$から$\phi^{+}$,角速度は$\dot{\phi}$から$\dot{\phi}^{+}$と変化するとする.
          本モデルの力積は\equref{Si}で計算され,系の運動量$p_i = \frac{\partial L}{\partial \dot{q}_i}$との関係は\equref{pi}である.
          \begin{eqnarray}
            \equlabel{Si}
            \hat{I}_i = \lim_{t^{+}\to t} \int_{t}^{t^{+}} S_i (\tau) \mathrm{d}\tau\\
            \equlabel{pi}
            p_i(t^{+})-p_i(t)=\hat{I}_i
            \end{eqnarray}
          本モデルにおいて,$\varphi$方向の運動量は保存されるため,\equref{impulse}が満たされる.
          \begin{eqnarray}
            \equlabel{impulse}
            M_{11}(l^{+})\dot{\varphi}^{+}-M_{11}(l)\dot{\varphi}=0
            \end{eqnarray}
          また,状態変数を
          \begin{eqnarray}
            \bm{x}=
            \begin{pmatrix}
              x_1\\
              x_2\\
              x_3\\
              x_4\\
              \end{pmatrix}
              =
              \begin{pmatrix}
                \varphi\\
                l\\
                \dot{\varphi}\\
                \dot{l}\\
                \end{pmatrix}
            \end{eqnarray}
          とすると,状態モデルは
          \begin{eqnarray}
            \dot{\bm{x}}=
            \begin{pmatrix}
              \dot{x}_1\\
              \dot{x}_2\\
              \dot{x}_3\\
              \dot{x}_4\\
              \end{pmatrix}
              =
              \begin{pmatrix}
                x_3\\
                x_4\\
                -M_{11}^{-1}(x_2)c_1(x_{1,2,3,4})\\
                -M_{22}^{-1}c_2(x_{1,2,3})+M_{22}^{-1}u\\
                \end{pmatrix}
            \end{eqnarray}
          となる.

          \subsection{励振シミュレーション}
          伸縮が時刻$t$秒から$t^{+}$秒の間に瞬間的に行われると仮定すると,
          伸縮後の状態$\bm{x}^{+}$は伸縮前の状態$\bm{x}$と\equref{impulse}を用いて
          \equref{xx}で表される.ここで,伸びる場合は縮む場合は$x_2^{+}=l_{\mathrm{min}}$,$x_2^{+}=l_{\mathrm{max}}$
          となり,\equref{l_change}の条件に基づいて変化させる.
          \begin{eqnarray}
            \equlabel{xx}
            \bm{x}^{+}&=&
            \begin{pmatrix}
                x_1\\
                x_2^{+}\\
                M_{11}^{-1}(x_2^{+})M_{11}(x_2)x_3\\
                0\\
                \end{pmatrix}\\
            \equlabel{l_change}
           l&=& \left\{
              \begin{aligned}
                l_{\mathrm{min}}\quad \mathrm{if}\,\varphi\dot{\varphi}>0\\
                l_{\mathrm{max}}\quad \mathrm{if}\,\varphi\dot{\varphi}<0
              \end{aligned}
            \right.
            \end{eqnarray}
          この瞬間的な伸縮により系の運動エネルギー$T$と位置エネルギー$U$は次のように変化する.
          \begin{eqnarray}
            \equlabel{dT}
            \Delta T_t^{t^{+}}&=&\left(\frac{M_{11}(x_2)}{M_{11}(x_2^{+})}-1\right)\frac{1}{2}M_{11}(x_2)x_3^2\\
            \equlabel{dU}
            \Delta U_t^{t^{+}}&=&gm(x_2^{+}-x_2)\cos{(x_1)}            
            \end{eqnarray}
          本実験で用いる実機に基づき,各パラメータと初期状態$(\varphi_{\mathrm{ini}},l_{\mathrm{ini}},\dot{\varphi}_{\mathrm{ini}},\dot{l}_{\mathrm{ini}})$
          を\tabref{parameter}としてルンゲクッタ法を用いてシミュレータを作成した.
          その結果を\figref{ExcitationSimulation.eps}に示す.
          横軸は時間$t$,縦軸はそれぞれ角度$\varphi$・角速度$\dot{\varphi}$・
          力学的エネルギー$E$・運動エネルギー$T$・ポテンシャルエネルギー$U$を示す.
          これにより,伸縮することによる重心移動でも励振が可能であることが確認できる.
          これらの内容はLieskovsk{\`y}らのおもりを動かすことによる励振の研究\cite{lieskovsky2023optimal}を
          伸縮機構に応用したものである.

          また,伸縮時の最大長$l_{\mathrm{max}}$を0.59 mから0.74 mまで3 cm刻みで
          変化させた場合の励振シミュレーションを\figref{changingLengthExcitation.eps}に示す.
          横軸は時間,縦軸は力学的エネルギーと角度を示している.
          シミュレーションでは,瞬間的に伸縮量$l$を変化させるために\equref{l_change}に基づいて直接伸縮量を上書きしている.
          そのため,シミュレーションにおける伸縮力$u$は瞬間的な伸縮が可能な力を想定している.
          このシミュレーション結果から,伸縮量を大きくするほど振幅増加量が大きくなり,伸縮量を小さくするほど伸縮増加量が小さくなることが分かる.
          ゆえに,伸縮量調整によって励振時の振幅操作が可能であると考えられる.
          なお,実際は伸縮量によって伸縮時間,加減速の影響が異なるが,それらの影響を含めた励振計測データから伸縮量調整法を検討した.
          \begin{table}[tb]
            \begin{center}
              \caption{Simulation parameters}
              \tablabel{parameter}
              \vspace{2mm}
              \begin{tabular}{c|c}
                \hline
                Variables & Values \\
                \hline
                $m$ [kg] & 3.0 \\
                $l_{\mathrm{min}}$ [m] & 0.56 \\
                $l_{\mathrm{max}}$ [m] & 0.74 \\
                $g$ $\mathrm{[m/s^2]}$ & 9.81 \\
                $\varphi_{\mathrm{ini}}$ [rad] & 0.3 \\
                $l_{\mathrm{ini}}$ [m]& $l_{\mathrm{max}}$ \\
                $\dot{\varphi}_{\mathrm{ini}}$ [rad/s] & 0.0 \\
                $\dot{l}_{\mathrm{ini}}$ [m/s] & 0.0 \\
                \hline
              \end{tabular}
            \end{center}
          \end{table}
          \fig{ExcitationSimulation.eps}{width=1\hsize}{Excitation simulation of extensible brachiation robot}
          \fig{changingLengthExcitation.eps}{width=1.0\hsize}{Excitation simulation with changing the max length}
        
          
        \newpage
        \section{振幅調整の流れ}

          本研究では振幅を基に伸縮量を操作し,振幅調整を行う.
          以下に振幅調整の流れを示す.
          \begin{enumerate}
            \item 目標振幅を決定
            \item IMUを用いてロボットの角度・角速度を計測
            \item 角速度の正負が入れ替わる時の角度を現在振幅とする
            \item 目標振幅に到達するまで現在振幅に基づく伸縮量操作を行う
            \item 目標振幅に到達したら伸縮量操作を終了
          \end{enumerate}

          \section{伸縮量操作}

          まず,半周期後の振幅の変化率と伸縮量の関係式を求める.
          重心が剛体中心にある剛体振子の運動方程式は通常,角度$\varphi$,長さ$l$,質量$m$,回転軸周りの慣性モーメント$J$,重力加速度$g$,減衰係数$c$を用いて
          \equref{equation}で表される.
          \begin{eqnarray}
            \equlabel{equation}
            J \frac{\mathrm{d}^2 \varphi}{\mathrm{d} t^2} +c \frac{\mathrm{d}\varphi}{\mathrm{d} t}+\frac{1}{2}mgl \sin{(\varphi)}=0
          \end{eqnarray}
          この運動方程式を用いて半周期後の振幅を求める場合,ルンゲ・クッタ法などを用いて数値的に解く必要があり時間がかかる.
          さらに,励振が含まれると右辺に励振のための外力項が加わるためより複雑になる.
          振子過程で半周期ごとに次の半周期後の振幅を調整するためにはより速く振幅を計算することが望まれる.
          そこで,本研究ではネイピア数を底とした指数関数近似式である\equref{Approximation Model1}に示す形式での近似モデル化を試みた.
          ここで現在振幅,振幅増加率,角振動数をそれぞれ$A_{\mathrm{now}}$,$\lambda$,$\omega$として
          半周期後の振幅$A_{\mathrm{next}}$を表す.
          \begin{eqnarray}
            \equlabel{Approximation Model1}
            A_{\mathrm{next}}=A_{\mathrm{now}}e^{\lambda \frac{\pi}{\omega}}
          \end{eqnarray}
          よって,現在振幅$A_{\mathrm{now}}$から半周期後に目標振幅$A_{\mathrm{ref}}$になるために必要な振幅増加率は\equref{lambda}で表される.
          \begin{eqnarray}
            \equlabel{lambda}
            \lambda=\frac{\omega}{\pi}\ln{\left(\frac{A_{\mathrm{ref}}}{A_{\mathrm{now}}}\right)}
          \end{eqnarray}
          また,各時刻において振動の近似を行うために,\equref{Approximation Model}の形で振動を表現した.
          なお,時刻$t$における変位を$\varphi(t)$で表す.
          \begin{eqnarray}
            \equlabel{Approximation Model}
            \varphi(t)=A_{\mathrm{now}}e^{\lambda t}\cos{(\omega t)}
          \end{eqnarray}

        \subsection{近似モデルの励振データへのフィッティング}
          
          実機の励振データの取得のために励振計測を行った.
          伸縮時の最小長は0.56 mで固定し,
          最大長のみ0.74 m,0.72 m,0.70 m,0.68 m,0.66 m,0.56 mと変化させた. 計測データ(角度・角速度)を\figref{excitationdata.eps}に示す.    
          \fig{excitationdata.eps}{width=1\hsize}{Excitation experiment data}
          なお,微小角近似ができない振幅になると周期は振幅に依存し,\equref{equation}において$\sin{(\varphi)}$が残り,非線形効果が現れる.
          そこで,近似モデルをフィッティングさせるために振幅と角振動数の関係式を求めた.
          最大長0.74 mでの励振データから,半周期ごとの経過時間$t_{\mathrm{half}}$ [s]と振幅$A$ [deg]を求め,
          \equref{T2omega}で示すように角振動数$\omega$を求め,\figref{OmegaAmp.eps}に示すようにプロットして線形近似を行った.
          \begin{eqnarray}
            \equlabel{T2omega}
            \omega=\frac{\pi}{t_{\mathrm{half}}}
          \end{eqnarray}
          \fig{OmegaAmp.eps}{width=0.6\hsize}{Relationship between angular frequency and amplitude}
          近似結果より,振幅と角振動数の関係は\equref{A2ome}となった.
          \begin{eqnarray}
            \equlabel{A2ome}
            \omega=-0.0126A+4.64
          \end{eqnarray}
          次に,最大長ごとに振幅増加率を調整し,計測データへの近似モデルのフィッティングを行った.
          振幅増加率はそれぞれ\tabref{ExcitationRate}とした.
          角度へのフィッティング結果を\figref{excitationfittingAngle.eps}に示す.
          % \newpage
          \begin{table}[b]
            \begin{center}
              \caption{Amplitude increase rate}
              \tablabel{ExcitationRate}
              \vspace{2mm}
              \begin{tabular}{c|c}
                \hline
                 $l_{\mathrm{max}}$ [m] & $\lambda$ [1/s] \\
                \hline
                0.74 &  0.132\\
                0.72 &  0.123\\
                0.70 &  0.115\\
                0.68 &  0.089\\
                0.66 &  0.025\\
                0.56 &  -0.02\\                       
                \hline
              \end{tabular}
            \end{center}
          \end{table}
          \fig{excitationfittingAngle.eps}{width=1\hsize}{Fitting results of angle}
          \fig{excitationfittingAngular.eps}{width=1\hsize}{Fitting results of angular velocity }
          また,\equref{Approximation Model}時間微分することにより,角速度は\equref{Approximation Angular velocity}となる.
          角速度データへのフィッティング結果を\figref{excitationfittingAngular.eps}に示す.
          \begin{eqnarray}
            \equlabel{Approximation Angular velocity}
            \dot{\varphi(t)}=A_{\mathrm{now}}e^{\lambda t}\{\lambda\cos{(\omega t)}-\omega\sin{(\omega t)}\}
          \end{eqnarray}
          \newpage
          フィッティング結果より,ロボット最大長が短くなるほど計測データとの誤差が生じている.
          原因として,ロボット最大長を0.74 mとしたときの計測データを基に角振動数と振幅の関係式を決定していることが挙げられる.
          ロボット最大長が異なることにより伸縮に要する時間が異なり,実際は角振動数と振幅の関係式もロボット最大長によって変わると考えられる.
          特にロボット最大長0.56 mにおけるフィッティング結果は,時間が経つにつれて大幅なずれが見られる.
          しかし,本研究では振動の半周期分の値のみを振幅調整に用いるため,適用できると考えた.

          次に\figref{RateLength.eps}に示すように,縦軸をロボット最大長,横軸を振幅増加率として\tabref{ExcitationRate}のデータをプロットし,線形近似した.
          \fig{RateLength.eps}{width=0.65\hsize}{Relationship between maximum link length and amplitude increase rate}
          近似式は\equref{length relation}となった.
          \begin{eqnarray}
            \equlabel{length relation}
            l_{\mathrm{max}}=0.99\lambda+0.60
          \end{eqnarray}
          \equref{lambda},\equref{A2ome}\equref{length relation}をまとめると\equref{l status}となり,
          半周期後に現在振幅から目標振幅に到達するために必要なロボット最大長$l_{\mathrm{max}}$は振幅から求めることができる.
          \begin{eqnarray}
            \equlabel{l status}
            l_{\mathrm{max}}=\frac{0.99(-0.0126A_{\mathrm{now}}+4.64)}{\pi}\ln{\left(\frac{A_{\mathrm{ref}}}{A_{\mathrm{now}}}\right)}+0.60
          \end{eqnarray}
        \newpage  
        \subsection{振幅調整実験}
        
        現在の振幅と目標振幅の値を基に伸縮量操作し,励振調整を行った.
        ここで,\equref{l status}において求められたロボット最大長がロボットの伸縮可能最大長0.74 mを超えた場合は最大長$l_{\mathrm{max}} =$0.74 m,
        伸縮可能最小長0.56 mを下回った場合は最大長$l_{\mathrm{max}} =$ 0.56 mとし,
        次の半周期で目標振幅への到達を試みる.
        なお,ロボット最大長をロボットの伸縮可能最小長0.56 mとすることにより,励振を含まない単純な減衰となる.
        実験では目標角度-120 deg,すなわち振子角度の負の範囲での目標振幅$A_{\mathrm{ref}} =$ 120 degとなるように振幅調整を行った.
        また,目標振幅との許容誤差を3 degとし,目標振幅到達後はロボット最大長指令値を$l_{\mathrm{max}} =$0.68 mとした.
        実験は計20回行った.振幅調整終了時の振幅データより,20回の実験の平均は119.37 deg,標準偏差は1.47であった.
        \figref{AdjustDataAnalysis.eps}に振幅調整終了時の振幅データを箱ひげ図として示す.赤色破線は平均値,緑色線は中央値を示す.
        目標振幅に対して,実験での平均値が近い値を示しており,さらに標準偏差が小さいためばらつきが少ないことが確認できた.
        目標振幅との誤差の平均は0.52 %,最大で2.19 %であり,最大誤差2.19 %でも振子過程における角速度の誤差は3 deg/s程であるため,
        後の章で述べるバーリリース条件到達への影響は小さいと考えられる.
        また,計測したデータのうち抜粋して2つの実験の角度と最大長指令値の時間変化データを\figref{Adjust41.eps}(実験1),\figref{Adjust42.eps}(実験2)に示す.
        なお,黒色一点鎖線は目標角度,赤色一点鎖線は目標振幅到達時刻を示す.
        振幅調整終了時の振幅はそれぞれ,実験1は118.1 deg,実験2は120.3 degであり,
        誤差はそれぞれ1.6 %,0.25 %であった.また,目標振幅との許容誤差をより小さくすることで,
        振幅調整精度を上げることができるが,目標振幅到達までの時間が長くなると考えられる.
        実験1は一回の伸縮量調整で目標振幅に到達したが,
        実験2では複数回の伸縮量調整が行われた.
        これは,実験2では計算上と実際の振幅変化の間のずれが大きいことを示しており,
        その原因としてグリッパーとバーの摩擦や,ロボットの構造上のずれやケーブルなどの干渉といった可能性が考えられる.
        一方で,複数回の伸縮量調整によって最終的には目標振幅に到達することができているため,
        ロバスト性が高いことが確認できた.
        \fig{AdjustDataAnalysis.eps}{width=0.7\hsize}{Amplitude adjustment experiment boxplot}
        \fig{Adjust41.eps}{width=0.7\hsize}{Amplitude adjustment experiment 1}
        \fig{Adjust42.eps}{width=0.7\hsize}{Amplitude adjustment experiment 2}
        





          



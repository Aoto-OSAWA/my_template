\chapter[空中過程を含まないブラキエーション動作の伸縮制御]%
{空中過程を含まないブラキエーション動作の\\伸縮制御}
        \section{はじめに}
        
        伸縮する単リンクブラキエーションロボットは,その「伸縮する」という特徴を活用することで,
        目標のバーまでの距離に合わせて伸縮させて移動することが可能となる.
        これにより,おもりを動かすことで励振するブラキエーションロボット\cite{akahane2022single}よりも
        自在な移動が実現できる.
        先行研究\cite{Hijiri:Robomech2024}では実験的に伸縮やグリッパーを閉じるタイミングを決定していた.
        しかし,伸縮のタイミングや伸縮量が適切ではない場合,バーを把持できないだけでなく,
        バーとロボットが衝突することで不具合が生じたり,破損してしまう.
        そこで,本章では空中過程を含まないブラキエーション動作における,バーの位置に基づいた伸縮制御について述べる.
          
        \section{伸縮タイミングの検討}

        伸縮して次のバーを把持する場合,伸縮するタイミングは\figref{noAerialBrachiationFig.eps}に示すように以下の4通りの方法が考えられる.

        \begin{itemize}
        \item 方法1 (\figref{noAerialBrachiationFig.eps}(1)):
        
        バーに対して下側から近づき,バーに近づいたらリンクを伸ばし始める.

        バー把持時のグリッパーに対する回転方向の負荷が小さいが,
        伸縮による角速度の変化や,伸縮時間を考慮する必要があり伸縮制御が複雑になる.
        
        \item 方法2 (\figref{noAerialBrachiationFig.eps}(2)):
        
        バーに対して上側から近づき,バーに近づいたらリンクを伸ばし始める.
        
        バー把持時のグリッパーに対する回転方向の負荷が小さいが,
        伸縮による角速度の変化や,伸縮時間を考慮する必要があり伸縮制御が複雑になる.

        \item 方法3 (\figref{noAerialBrachiationFig.eps}(3)):
        
        最高点においてリンクを目標の長さまで伸ばし終えた状態で上側からバーに近づく.
        
        振子の最高点に達したときに伸縮するため,伸縮制御が容易であるが,
        最高点が目標のバーから離れている場合,バー把持時のグリッパーに対する回転方向の負荷が大きくなる.
        
        \item 方法4 (\figref{noAerialBrachiationFig.eps}(4)):
        
        最高点においてリンクを目標の長さまで伸ばし終えた状態で下側からバーに近づく.

        方法3と同様に伸縮制御が容易であるが,
        把持していたバー(\figref{noAerialBrachiationFig.eps}におけるbar 1)と衝突する可能性がある.

        \end{itemize}

        以上のうち,方法3は振子の最高点が目標のバーから離れすぎない位置まで励振したタイミングでリンクを伸ばすことで,
        バー把持時のグリッパーに対する回転方向の負荷を小さくすることができるため,本研究では方法3を採用した.


        \fig{noAerialBrachiationFig.eps}{width=1.0\hsize}{Schematic Diagram}

        \section{}



        ここにバー座標の図、座標から距離、角度、移動可能な範囲を半円で示したり(角速度はいらないかな)

        その次に上から掴むか、下からか、どこで伸ばすかの議論。




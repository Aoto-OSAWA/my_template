\chapter[結論および今後の展望]%
{結論および今後の展望}
        \section{結論}

        本研究では,伸縮する単リンクブラキエーションロボットが2次元空間を自在に移動することを目標とし,
        目標バーの位置に基づいたブラキエーション動作を行うためのシステムを提案・検証・考察した.
        目標バーの位置に基づいて,空中過程を含まないブラキエーションでは伸縮タイミングを,空中過程を含むブラキエーションでは
        バーリリース条件を決定し,振子過程で実現するためには振幅調整が望まれる.
        そこで,本研究では伸縮する機構を活かした振幅調整法を提案し,実機を用いて振幅調整実験を行い,振幅調整法の有用性を確認した.
        また,実機を用いて空中過程を含まないブラキエーション動作において,目標バーの位置に基づいて目標振幅を決定し,
        提案した振幅調整法を基に目標振幅を実現し,目標バーの把持に成功した.
        さらに,振幅調整法を用いて最適なバーリリースを実現し,空中過程を含む目標バーへの到達と目標バーの把持に成功した.
       

          

        \section{今後の展望}
        
        今後の展望として,移動可能範囲の拡張やロバスト性向上などがある.
        本研究では目標とするバーを掴むグリッパーを片方のみに限定して研究を行ったが,移動前に最初のバーを把持しているグリッパーの空中過程での軌道を考慮することで,
        把持可能な範囲が広がる可能性がある.
        また,3次元方向に移動することができる機構にすることにより,本研究で行った2次元方向の移動よりも自在な移動が可能になる.
        さらに,目標とするバーを認識する機構などを用いることで,より正確な把持によるロバスト性向上やバーの位置を認識して自律的に移動することも期待できる.
        加えて,本研究のブラキエーション実験では振子過程においてグリッパーの形状により,振り子の回転軸方向に振動が発生してしまい,
        それにより振幅調整の誤差や把持時のグリッパー方向のずれが生じてしまった.
        ゆえに,振り子の回転軸方向の振動が発生しないグリッパーの形状の検討も望まれる.
        また,本研究で提案した最適なバーリリース条件でリリースした場合でも,理想的な軌道からずれてしまう可能性がある.
        そのため,空中過程でも伸縮制御を行うなどして軌道修正をすることでロバスト性を向上させられると考える.
